\divider
Tarn and Peter had been traveling for a few days, and were now settling down for their fourth night camped off the road.  By now Tarn was getting used to the traveler's life: he knew how to find firewood, what sort and how much to collect; he could hunt with some success; and with each passing night he was getting to sleep easier.

On this night, however, it began to rain, which was a phenomenon to which Tarn was very much unaccustomed.  They were camped underneath some trees to give their fire some degree of shelter from the wind, and they had fixed a sheet to the trunks, covering the ground, so that they could sleep without getting too wet.  Even so, it was miserable and cold, with water getting everywhere, and Tarn knew that he would never get to sleep.  He glanced over and saw Peter lying with his eyes closed, but couldn't be sure whether he was asleep, or merely trying to will himself so.

A flash of lightning lit up the sky, and suddenly Tarn heard intense, excited growling, from all around them.  He jumped up, hammer in hand, and yelled to Peter to wake up.  He glanced around in the dim light, and saw the fire's weak flickering reflected in three, four \ldots\  at least five monstrous faces, surrounding the camp.

``Goblins!'' Peter yelled.  The creatures were small, shorter than Tarn and much thinner.  They had pale greyish skin, long ears and noses, and wrinkled, sneering faces.  They wore crude clothing: tattered rags, with the odd leather jerkin or pauldrons that were presumably looted from some other unfortunate travelers.  They held clubs and daggers at the ready.

Having lost the element of surprise, the goblins began to advance slowly.

``Do you know how to fight?'' Tarn shouted over the sound of the rain.

``No!  I've never needed to!'' came Peter's reply.

``Hold your stick close, and hit any that come near!''

Tarn stared at the goblins, slowly crouching down to pick up his shield.  He got up suddenly, and at that, three of them lunged at Tarn.  He held up the shield to deflect the initial blows, then lowered it slightly so that he could see.  One goblin remained in front of him; the other two had retreated after their first blows.  Tarn swung his hammer, and sent the goblin flying, dead.

He glanced over at Peter.  Two goblins circled around him, with the poor cleric holding up his stick as if it would ward them off.  Tarn yelled ``Come at me, you ugly things!'', and one of them peeled off to pursue Tarn instead.  At least now Peter had only one to deal with.

While Tarn was distracted, one of his original attackers had approached him again and struck Tarn in his stomach.  ``Argh!'' he grunted, an intense stabbing pain rippling through his abdomen.  He swung at the goblin, but it had already retreated out of range once again.  Tarn glanced down at his gut.  He dropped his shield so that he could hold the wound, and grimaced with pain.

Tarn was accustomed to fighting wild beasts, and even other dwarves, but these goblins were something different.  They were unceasing, rabid, ferocious; and yet crafty.  And now that he had a few moments of respite while the goblins kept their distance, he begain to panic.

\emph{What am I doing here?}  The weather was alien, with water getting everywhere, and noise and fog obscuring his senses.  They were beset by enemies he had never known before, and everything they did surprised him.  He was not prepared for any of this, and that made this a very dangerous situation.  He did not expect the world outside his mountain to be so wild, so messy, so dangerous.  There was such chaos out here under the open sky; such bloodlust and violence and noise.  He forgot the elves and the sword, and the water supply.  All he could think about was that it was a mistake to leave the mountain; that this was not who he was.

Tarn took a deep breath and straightened his back: there would be time for such thinking later.  There was still a goblin trying to attack Peter, barely being held at bay by his walking stick---if anything the goblin was toying with him.  Tarn rushed towards it, hammer raised, and crushed it in one swing.  The others, which had begun to follow him, instead turned to run.

Tarn ran after them, one hand holding his raised hammer and the other clutching his wounded stomach.  Peter yelled after him, chasing and imploring him to give up the chase.  Soon enough, Tarn tripped over something in the wet dark, and struggled to get back up.  Peter caught up to him, turned him onto his back, and observed the wound.

The dwarf tried to tell Peter about the healing herbs back at the camp, but he couldn't get the words out, between the exhausted panting and the pain.  Soon he gave up trying to talk.  Peter held both of his hands over Tarn's abdomen, closed his eyes, and began to mutter something, some kind of chant.  A flash of light seemed to emanate from Peter's hands.  Not lightning, but wider, warmer, more sustained.  Like a pulse of sunlight.  And with it, the pain suddenly stopped.  Tarn looked down and saw that his wound was gone.  He could stand up.

Tarn stared at Peter, who in return only smiled mildly.  They began trudging back.