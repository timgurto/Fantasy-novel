\chapter{Comforts}

Tarn and Peter returned to their campsite, and were horrified at what they found there.  The hrose and the pony both lay slaughtered.  Their money was stolen, as well as their food, water skins, and the healing herbs that the librarian had given to Tarn.  Their bed rolls were slashed.

Tarn searched the ground nearby, and found his shield.  Nearby, Peter found the dagger that one of the goblins had dropped.  He held it up and stared at it with wide eyes.

``I've never been so scared in my life,'' Peter said.

Tarn pointed at the dagger.  ``I'd hold onto that if I were you.  Might come in handy.''

``You mean, if we get attacked again?''  He stared into the distance.

The dwarf grabbed his arm.  ``Come on; let's find some shelter.''

Tired, wet and injured, they stumbled through the woods.  They had nothing else now: no transport, no money, nowhere to sleep.  But they could sort that mess out in the morning.  Right now they needed to find shelter from the rain, and somewhere to rest.

They found no large trees, no caves, no rocky outcrops.  Eventually Tarn spotted a short ridge.

``Let's rest there,'' he said.  ``It won't keep the rain off, but at least we can sit up out of the mud.''

Peter sat, leaning against the ridge, and looked like he was about to cry.  Tarn grabbed a handfull of dirt, then another, until he'd made a shallow cavity.  He then slumped into it, resting beside Peter, his back embraced by the wet dirt.  The rain continued pouring down on them.  But at least now they could finally rest.

``Thank you for what you did out there, to my stab wound,'' he said to Peter.

``It was the Light that healed you.  I was just its agent.''

``That's what you do as a cleric?''

``Yes,'' said the man slowly, continuing to stare forward.  ``I speak to the Light, and ask for its intervention.  Part of my training is knowing when it will help, and when it won't.''

``If it's a good god, why wouldn't he \emph{always} help?''

Peter sighed deeply.  He was interested in theology and generally didn't mind discussing it, but his mind was still reeling from the events of the evening.

He continued.  ``The Light desires the world to be good.  If I can make a convincing case that some intervention will help to move the world in that direction, then it lends its assistance.''

``And this Light wanted to ease my pain?''

``Something like that.  The Light elevates things to their full potential.  Your well-being, the balance within your body, is good.  The goblin that stabbed you was undoing something good; creating disorder.  The Light helped to fix that, to restore the state of the world.''

Tarn was surprised to hear that there was some method to this religion, and not just random incantations and superstition.  Then again, he was surprised to actually have been healed, magically, from what could have been a mortal wound.

``I always thought religion was about following rules and doing the right thing.  Maybe the dwarves in my home city aren't doing it right.''

``Are you from Orehome?'' Peter asked.  Tarn nodded.  ``It's about morals too.  As I said, the Light has a preference for the world.  What it can't control directly is the free choices that a person makes.  And so it's up to us to act in the right way.''

``And I can trust some cleric to come up with the correct rules?'' Tarn asked, quickly adding ''meaning no offense.''

``None taken,'' Peter said.  He seemed a little more engaged now.  Maybe the discussion was distracting him from the night's chaos.

``Try to follow your conscience---it's generally a good guide.  That's what the necklace is for, to remind you to make the right choices.''

Tarn considered this, fingering his necklace.  ``Did those goblins have consciences?''

``I believe they did.  The sentient races are the only beings who can choose their actions, and goblins are sentient.  They just seem to have fallen somewhat, become animalistic and instinctive.  Wild.  Some clerics have tried to convert goblins in the past---it never works, and they don't return.  But there is always hope \ldots'' Peter trailed off, wistfully.  Then, ``after tonight, I have a newfound respect for those missionaries.''

``You did well defending yourself tonight,'' Tarn offered, as cheerfully as he could.

``I wish I could have been more help to you.''

``I got stabbed and you healed me.  What more help could a dwarf want?  Anyway, you have a dagger now.  You can start practicing with it,'' he joked.

Peter stared at the dagger in his hand.  It was an instrument of death.  A symbol of danger.  When he looked at it, he felt that same fear again, that feeling of helplessness as violent creatures surrounded him with greedy bloodlust in their eyes.  He couldn't imagine actually using it.

With the rain continuing to pour down, the pair tried to sleep.  Tarn closed his eyes, leaning into the impression he'd made in the dirt, enjoying the feeling of contact against his back.  It was a shelter that he himself had carved out.  However shallow, however exposed to the rain.  It gave him a primal feeling of satisfaction, of agency in the midst of this mess.  An old song drifted into his memory, a children's song well-known to most dwarves, and he absent-mindedly began to sing it to himself.

\settowidth{\versewidth}{When the world was just born it was all sea and sky;}
\begin{verse}[\versewidth]\poem{The Song of Giants}
When the world was just born it was all sea and sky;\\
there was stone, but it all on the seabed did lie.\\
Then the Giants, in sculpting their caves, piled it high\\
and created the land where a soul could be dry.

But some giants were greedy, and split from the brood\\
as they coveted stone for their dwellings and food,\\
so they rose from the sea, and the land they subdued,\\
they took strides with their feet, and they found shelters crude.

But they longed for their caves, to be girded with stone,\\
while they warred with each other: in holes they lay prone,\\
hulking walls were amassed, giant boulders were thrown,\\
and in this way the caves and the mountains were grown.

Over time they diminished, and now in their place\\
are the Dwarves: not as big, but a filial race;\\
they possess the same drive to carve out their own space,\\
and in conquering stone may they too find their grace.
\end{verse}






Tarn slept.


``I don't know how we're going to get to Westport now,'' Peter said glumly.

``We could perhaps head back to the road in the morning, and wait for a cart?`` offered Tarn.

``That could take days.''  Then, the cleric asked ``where are these elves of yours?''

``They live in a swamp, apparently.  I was told that sailing down the coast would be the best way to get there.''

``We're about as close to the coast now as we are to Westport.  Might it be worth trying 

Tarn decided that it was worth telling Peter about his quest.