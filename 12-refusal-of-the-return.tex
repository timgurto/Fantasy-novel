As they walked back towards the elemental's chamber, the words of \mothzam\ \driktur\ echoed in his ears.  \emph{Was it actually possible to bring the city back?}  He understood that the city itself was dead and could never live again, but what about its treasures?  The relics, and crafts, and maybe even lost knowledge?  Reason clearly rejected that possibility, but his heart compelled him to dream of it.  He imagined the city being resurrected, even if it were mere ruins.  He imagined being the one to have done so.  And the power to do it might be right here, within these tunnels.

He had just killed the person planning to do that very thing.  He had also just lost a friend, who died to prevent it from happening.  Yet here he was, still considering it.  After all he had been through, why did this conflict persist?

When Tarn and Peter reached the chamber, the dwarf felt compelled to ask the question that he was never quite satisfied had been answered: ``you have said that you would not restore \valdunmir.  But regardless of that choice, is it actually possible to do so?''

The elemental's deep gurgling voice replied, ``No.  A thousand of my kind  could not exert enough influence over the seas to lift an entire city from the seabed.  And even if we could, young dwarf, what then?  It would take sustained effort to keep it afloat.  I know not how it was held up when originally built, but that spell was broken long ago.  I would not inflict that burden on my people.  I would die before allowing it.''

``I understand,'' said Tarn, both disappointed that \mothzam's dream truly was mere madness, and relieved that the decision was not in his power to make.  ``May I ask you about the sword that purified you?''

``Of course you may.  You provided a great service to me with that blade, and I would gladly assist you if I can.''

``I believe it is the metal itself that has the special ability to purify water, rather than it being some property of how the sword was made.  You have felt its power; is that how it seems to you?''

``It is.  The metal of the sword displaced the filth that had accummulated in my body.  All of that was pushed out, while I remain whole.  The metal that accomplished this was similarly unaffected.''

``The metal was unchanged?''

``That is correct.''

``That's an astounding power!  It means the sword could continue working indefinitely!''

``It is indeed powerful,'' said the elemental, ``but it is also rare.  In my thousands of years of travel, I have not seen this metal anywhere else in the world, in any form, or even as an ore below the sea.''

Tarn stared at the sword.  Could this really be all of the metal that existed?

``I have one more question, if you don't mind.  I was driven on this quest by my city, a mountain fortress whose supply of drinking water has become tainted.  Those who drink it get sick in the stomach.  Do you know if this metal can purify that water, that corruption?''

The elemental sat in thought for a minute before answering.  ``I know the problem of which you speak.  It is caused by tiny creatures in the water, interlopers that make ill anybody with a stomach.  And I am quite sure that your sword will push those creatures out of the water, just as it did with my dirt and mould and algae.''

A sense of relief washed over Tarn.  After all this time, this uncertainty, he now had a concrete answer: Korbarthrond's problems could be solved, and he had found the solution.

Tarn produced the bottle from his pocket, the one taken from \mothzam's staff.  ``My essence!'' the elemental exclaimed upon seeing it.  ``Then is \mothzam\ \driktur\ slain?''

``He is,'' replied Tarn.  ``What do I need to do in order to free you?''

``If you merely open the bottle and give it to me, I can take the essence back into my being.''  Tarn did so, and the elemental immediately erupted in excitement.  It twisted and churned much faster than usual, spilling more of itself and picking it all back up again.

``Thank you, strangers.  My torment is finally over!''

Tarn and Peter couldn't help but smile at the display of joy, especially for a creature that had until now behaved so reservedly.





