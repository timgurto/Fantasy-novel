\chapter{Kobarthrond}
Tarn, son of Rolg, stood straight and still, his eyes peering over the city's entrance hall one last time before he ended his shift.  Guarding the city was as uneventful today as it usually was; the worst invaders Tarn had ever needed to repel from the city were bears and other wild animals.  In a more turbulent time he may have been part of an army marching off to fight for some great cause, but in this age of peace he stood \emph{inside} the gates, looking inwards.  Not that he minded---he was a Dwarf, after all, and like most of his kind he was most comfortable nestled within the carved bosom of his mountain.

The entrance hall was cavernous and showy.  Wide columns stretched to the ceiling, so high that it was hard to see their tops among the distant dark.  The floor of the hall was tiled with polished marble, tapping sharply with each footstep from the Dwarves going about their business.  Because it was so close to the main gate, over the generations this hall had developed into a marketplace for dealing with visitors from outside.  The city received only one or two traders per day, and so most of the merchants here still served the local Dwarves.  Occupying the stall nearest to Tarn was a silversmith, whose tables were decorated with scales and stacks of coins and ingots.  Silver was the city's main export, and it was mined and smelted here in the mountain.  In the stall next to that one was a wood carver, selling ornaments.  There were larger markets deeper within the city, but this was the place to find those goods made to appeal to outside buyers.

The stone walls were engraved with elaborate patterns and images.  These walls were carved in-situ, straight into the original rock, and not placed there.  Marbled throughout them were veins of a pale blue mineral which the Dwarves named simply Omunkor, or ``Blue-ore'' in the language of Men.  Where the blue touched the engravings, it was polished to be bright and clear, so as to be more easily seen.  Omunkor permeated much of the city, and the Dwarves generally kept it intact wherever found: they could find no functional use for it, only decorative, and it had become a source of pride for them.  Thus was the city itself called Kobarthrond---``Orehome''.

A large inscription was engraved prominently on one of the high walls flanking the entrance hall, a message to citizens and visitors alike.  Its glyphs consisted of the hard, straight marks of the Dwarven script, adapted to be carved into hard materials.  The inscription read:
\dwarvenInscription{%
bamk azthu thaku trobu gi brolzolg\\%
dult kob krithsulb gorg izun gi kugzolg.}
Roughly translated, it means:
\begin{verse}\poem{The exhortation of Kobarthrond}
Like ore, innate and polished in our walls,\\
should you be true yet shine within these halls.
\end{verse}

Tarn always enjoyed patrolling here, because it gave him an opportunity to admire the craftsmanship of those wall engravings.  The images depicted various stories from the history and myth of the city and her people, and they exhibited the care and love of fine work that most Dwarves applied to their various vocations.  Guardwork afforded little opportunity for Tarn himself to scratch this itch, and so he instead found opportunities to appreciate the work of others.

His replacement arrived just as the deep bell announced shift's end.

``All quiet today,'' said Tarn.

The other guard smiled and nodded, and Tarn began to head home, while the others in the market area mostly stayed put.  Craftsmen and merchants worked on their own time; it was only the city workers like Tarn who followed the shift system.  He walked home faster than usual, as he had arranged to meet a long-time friend of his after work today: Lawrence, a Human trader who was visiting the city.

Following a number of corridors and common areas, Tarn reached his quarters, a modest apartment carved into the mountain.
Tarn's admiration for fine work extended beyond enjoying the city's commons, and into into his home.  He maintained a collection of personal treasures: gold rings embedded with brightly coloured gemstones, and small figures of polished silver and carved stone.  Taking pride of place on a shelf near his bed was a scale model of the mountain, about the size of his fist and carved from a solid piece of Omunkor.

The bulk of Tarn's wealth lay in a cache of ingots and coins made from gold and silver, which he loved for their precision, detail and shine almost as much as for their value.  The silver was mined here in Kobarthrond, but the gold needed to be imported as the mountain had none beneath it.  Most of the coins were struck here though, as Kobarthrond, like most every other Dwarven city, took advantage of every opportunity to make its mark.  That being said, Tarn did have a few gold coins from other cities, as they sparked a romantic fantasy of the wider world, and of Dwarves spread far abroad yet engaging in the same pursuits and industries that they enjoyed here.  While Tarn had no interest in actually \emph{seeing} those far-off cities, he felt reassured---and proud---to be a part of something larger.

Tarn changed out of his uniform, and headed to the tavern nearest his apartment.  He found Lawrence already there, at a table with two mugs of beer in front of him.  As with all Men, Lawrence was tall and lanky compared with most Dwarves, with a small nose and shallow eyes.  Seeing Tarn, he stood up with his arms outstretched.

``Tarn, my old friend!  How are you?''

``Good, good.  And you?  How was the road?'' Tarn replied, sitting.

Lawrence lived in Silverdale, the town of Men in the valley below the mountain and the closest major settlement.  Silverdale was built on the Kobarthrond River, which flowed east from the mountain towards the sea.  Lawrence didn't sail up the river, though: although it was wide, it meandered through a thick, treacherous forest---called Riverwood by the Men---that had claimed many ships.  And so, while they could engage in water trade downstream, no trader came to Kobarthrond except by road and through the main gate.

``The days were quiet and the nights were mild.  All a man can ask for,'' he replied.


Tarn leaned forward.  ``Anything new in town?  What are the Men up to these days?''  Men were always coming up with new designs, theories, and technologies.  Many were amusing failures, but sometimes real innovations took place.  On his last visit, Lawrence had told him about an alchemist who had accidentally created a new kind of medicine!

``Nothing much in Silverdale, but I did hear that Westport is experimenting with new kinds of fertiliser.  If it works, they think they can improve crop yields by a lot.''  Westport was a major Human city, the most influential power in the region.

The two old friends continued talking about their respective cities and peoples, and exchanging jokes and stories, and soon they finished their drinks.

Emptying his drink, Tarn put his mug down and wiped his beard with the back of his hand.  Lawrence had no beard at all to match his dark-red hair, though Tarn understood facial hair to be less common among Men, and a matter of personal style.  Dwarves, on the other hand, grew their beards long by convention, braiding and decorating them with care.  Seeing anyone clean shaven, even a Man, and even a familiar Man like Lawrence, still felt odd even after many years of friendship.

``Can I buy you another?'' Tarn asked.  Then, jokingly, ``or would you prefer water instead of beer?''

``Just because I don't bathe in the stuff like a Dwarf, doesn't mean I can't hold my own!''

Dwarves drank beer almost as much as they did water, and on social occasions like this there was no excuse to drink anything else.

``Anyway, I already tried ordering some,'' Lawrence continued.  ``The bartender said they were out.''

``Out of water?''

``And not just today.  He said they'd been having trouble for days.''

``Odd,'' said Tarn, trying to remember the last time he'd replenished his own water barrel.

The city had one primary well near the centre, going deep into the aquifer.  Other, smaller wells were connected to it.  Tarn had never known any of the wells to go dry.

Lawrence went on.  ``In fact, that's exactly what brings me to the city this time.  Your king ordered a shipment of water from the river, and I just carted in eight full barrels of it.''

If there was a problem with the city's water supply, Tarn wanted to know about it.  He may not have been able to do much about that sort of problem, but he felt some level of responsibility over the city---perhaps an inclination that came with his position as a guard---and wanted to keep on top of issues like this.  So he resolved to get some answers.

After another round of beer, Tarn and Lawrence said their goodbyes. Lawrence headed to the inn and stables near the front gate, where he had a room rented and his cart was interred.  Tarn headed towards the heart of the mountain, to visit the king and ask him what was going on.

\chapter{The King's Request}

Two uniformed guards stood in front of the throne room.  It was blocked by a large door of dark wood, banded at the top and bottom with iron engraved in a complex geometric pattern.  In the middle of the door was an outline of their mountain, carved into the wood and inlaid with bright silver wire.  The guards smiled as they recognised their peer.

``Guardsman Tarn!  What brings you to the throne room?''

``Hello boys.  I'd like an audience with His Majesty.''

One of the guards muttered something through the door to someone on the other side, and received a low, muffled response.  He told Tarn to wait for a moment.  The three guards chatted for a few minutes, until eventually the low voice spoke again from the other side of the door.

``You can go ahead in, Tarn,'' said the guard.  He pulled the handle and the door swung open.

Athzad, son of Valkold, was the king of Kobarthrond.  He had reigned for nearly twenty years, and was well-regarded by the citizens of the mountain.  King Athzad had a long, thick, brown beard, split into three with silver thread braided into each part.  He wore a crown on his head, a band of patterned gold decorated with many jewels, uniquely coloured but all cut to the same size and shape, brightly reflecting the flickering light from the throne room's torches.  The throne beneath him was solid stone, carved in precise straight angles and rippled with polished Omunkor.

Tarn entered the room, approached the throne, and bowed.  The guard standing next to the throne stared straight ahead.

``What can I do for you, guardsman?'' King Athzad asked.

``Your Majesty, I have heard that the city is having trouble with its water supply.  I want to know if it's true, and if possible, the cause.''

The king sighed.  ``You heard correct, though this is not publicly known, and I ask you not to spread it around and cause a panic.

``For about two weeks now, Dwarves have been getting sick from our wells.  Something goes wrong in the gut.  We don't know what's causing it.''

``So that's why we're importing water?''

``That's right,'' the king answered.  ``The only water provided for drinking is what we can get from outside.  The wells are restricted to industrial uses, washing and brewing.''

``Brewing?  Is our beer being poisoned?'' Tarn snapped quickly, in a tone not fit for the throne room.  The guard by the throne raised an eyebrow and tightened his grip on his spear.

The king raised an eyebrow at Tarn, but maintained his steady voice.  ``I understand your concern.  Boiling the water appears to make it safe, and so our beer is not dangerous.''

Tarn took a deep breath.  ``I apologise, Your Majesty.  Is there anything we can do about it?'' he asked.

``We are pursuing a number of strategies,'' came the reply.  ``One team is exploring the darker caves and tunnels for potential new sources.  Another is engaged in fetching water from the river outside.  And we will continue importing what we need until a solution is found.''

Tarn was not optimistic.  Water sources within a mountain are rare; and anything truly accessible would have been found by now.  Fetching water from the river, through that forest, was too labour-intensive.  And long-term, buying water seemed like economic suicide.  But he held his tongue, and took care to get his thoughts in order before speaking.  He thought about Lawrence's stories about Men, and their experiments and advances.

``Your majesty,'' he began slowly, ``I think it's worth sending somebody to the Human town downriver, to see if they know of a solution.''  Tarn was careful not to directly criticise the king's other strategies, or to suggest that Men had any kind of superiority over Dwarves.
``Men lack our sense of beauty and accomplishment, and for want of a similar greatness they constantly try new things and push new boundaries with plants and animals and machinery.  They may have a technology or a medicine that we do not.''

King Athzad considered this silently for a moment, before responding, ``Very well.  If you believe the Men of Silverdale possess some secret that will save Kobarthrond, then you will be the one to go there, and determine for yourself whether they have anything useful''.

``Me?'' asked Tarn, blinking.

``With your affinity for the Men, you are the best placed to find the ways in which they can help us,'' replied the king with an almost imperceptible hint of sarcasm.

The decision had been made.  Tarn thanked the king, bowed, and took his leave.