\chapter{Blah}


was a fortress cut deep into a mountain.  Throk's race of mountain Dwarves had many such cities, and Kobarthrond was moderately sized among those.  It was named for the numerous metal ores that were discovered within the mountain, still evident in the rippling bands of brown and grey-green that decorated many of the city's stone walls and cavernous ceilings.  Those particular metals were not worth the expenditure to mine and process, and in some cases the Dwarves simply knew of no practical use for them.  Instead they stood as a reminder of their beloved metallurgical industries, and a testament to the Dwarven affinity for being underground.



Tarv stood straight and still, his eyes peering over the city's main entrance hall one last time before he ended his shift.  Guarding the city was usually uneventful; the only invaders Tarv had ever needed to repel from the city were bears and other wild animals.  In a more turbulent time he may have been part of an army marching off to fight for some great cause, but in this age of peace he stood inside the gates looking inwards.  Not that he minded---he was a Dwarf, after all, and like most of his kind he was most comfortable nestled in the carved bosom of his mountain.

It was a large hall, cavernous and showy.  Wide columns stretched to the ceiling, so high that it was hard to see the tops in the distant dark.  The floor of the hall was polished marble, tapping sharply with each footstep from the Dwarves going about their business.  Because it was so close to the main gate, over the generations this hall had developed into a marketplace for dealing with visitors from outside.  The city received only one or two traders per day, and so most of the merchants here still served the local Dwarves.  Occupying the stall nearest to Tarv was a silversmith, whose tables were decorated with scales and stacks of coins and ingots.  Silver was the city's main export, and it was mined and smelted here in the mountain.  In the stall next to that one was a wood carver, selling ornaments.  There were larger markets deeper within the city, but this was the place to find the goods made to appeal to outside buyers.

The stone walls were engraved with elaborate patterns and images of the histories of the city and its people.  These walls were carved straight into the rock, not constructed, and rippled throughout them were veins of a pale blue mineral.  Where the blue touched the engravings, it was polished to be bright and clear, much more easily visible.  This mineral permeated much of the city, and the Dwarves generally kept it intact: they could find no functional use for it, only decorative, and it had become a source of pride for them.  Thus was the city called Kobarthrond, or ``Orehome" in the language of Men.  A large message was engraved prominently on one of the high walls flanking the entrance hall.  Its glyphs, in the Dwarven script, consisted of hard, straight marks suited to being carved into hard materials.  The message read:
\begin{verse}
\dwarven{bamk azthu thaku trobu gi brolzolg}\\
\dwarven{dult kob krithsulb gorg izun gi kugzolg.}
\end{verse}
Roughly translated, it means:
\begin{verse}
Like ore, innate yet polished in our walls,\\
should you be true yet shine within these halls.
\end{verse}