\chapter{Kobarthrond}
Tarn stood straight and still, his eyes peering over the city's entrance hall one last time before he ended his shift.  Guarding the city was usually uneventful; the only invaders Tarn had ever needed to repel from the city were bears and other wild animals.  In a more turbulent time he may have been part of an army marching off to fight for some great cause, but in this age of peace he stood \emph{inside} the gates, looking inwards.  Not that he minded---he was a Dwarf, after all, and like most of his kind he was most comfortable nestled within the carved bosom of his mountain.

The entrance hall was cavernous and showy.  Wide columns stretched to the ceiling, so high that it was hard to see their tops among the distant dark.  The floor of the hall was tiled with polished marble, tapping sharply with each footstep from the Dwarves going about their business.  Because it was so close to the main gate, over the generations this hall had developed into a marketplace for dealing with visitors from outside.  The city received only one or two traders per day, and so most of the merchants here still served the local Dwarves.  Occupying the stall nearest to Tarn was a silversmith, whose tables were decorated with scales and stacks of coins and ingots.  Silver was the city's main export, and it was mined and smelted here in the mountain.  In the stall next to that one was a wood carver, selling ornaments.  There were larger markets deeper within the city, but this was the place to find those goods made to appeal to outside buyers.

The stone walls were engraved with elaborate patterns and images.  These walls were carved in-situ, straight into the original rock, and not placed there.  Rippled throughout them were veins of a pale blue mineral which the Dwarves named simply \emph{Omunkor}, or ``Blue-ore" in the language of Men.  Where the blue touched the engravings, it was polished to be bright and clear, so as to be more easily seen.  Omunkor permeated much of the city, and the Dwarves generally kept it intact wherever found: they could find no functional use for it, only decorative, and it had become a source of pride for them.  Thus was the city itself called Kobarthrond, ``Orehome".

A large inscription was engraved prominently on one of the high walls flanking the entrance hall, a message to citizens and visitors alike.  Its glyphs consisted of the hard, straight marks of the Dwarven script, adapted to be carved into hard materials.  The inscription read:
\begin{verse}
\dwarven{bamk azthu thaku trobu gi brolzolg}\\
\dwarven{dult kob krithsulb gorg izun gi kugzolg.}
\end{verse}
Roughly translated, it means:
\begin{verse}
Like ore, innate and polished in our walls,\\
should you be true yet shine within these halls.
\end{verse}

Tarn always enjoyed patrolling here, because it gave him an opportunity to admire the craftsmanship of those wall engravings.  The images depicted various stories from the history and myth of the city and her people, and they exhibited the care and love of fine work that most Dwarves applied to their various vocations.  Guardwork afforded little opportunity for Tarn to scratch this itch, and so he instead appreciated the work of others.

His replacement arrived just as the deep bell announced shift's end.

``All quiet today", said Tarn.

The other guard smiled and nodded, and Tarn began to head home, while the others in the market area mostly stayed put.  Craftsmen and merchants worked on their own time; it was only the city workers like Tarn who followed the shift system.  He walked home faster than usual, as he had arranged to meet a long-time friend of his after work today: Lawrence, a Human trader who was visiting the city.

Following a number of corridors and common areas, Tarn reached his quarters, a modest apartment carved into the mountain.
Tarn's admiration for fine work extended beyond enjoying the city's commons, and into into his home.  He maintained a collection of personal treasures: gold rings embedded with brightly coloured gemstones, and small figures of polished silver and carved stone.  The bulk of his wealth lay in a cache of ingots and coins made from gold and silver, which he loved for their precision, detail and shine almost as much as for their value.  Taking pride of place on a shelf near Tarn's bed was a scale model of the mountain, about the size of his fist and carved from a solid piece of Omunkor.

Tarn changed out of his uniform, and headed to the tavern nearest his apartment.  He found Lawrence already there, at a table with two mugs of beer in front of him.  As with all Men, Lawrence was tall and lanky compared with most Dwarves, with a small nose and shallow eyes.  Seeing Tarn, he stood up with his arms outstretched.

``Tarn, my old friend!  How are you?" he said cheerfully.

``Good, good.  And you?  How was the road?"

Lawrence lived in Hamlet, the town of Men in the valley below the mountain and the closest major settlement.  Hamlet was built on the Kobarthrond River, which flowed east from the mountain towards the sea.  Lawrence didn't sail up the river, though: although it was wide, it meandered through a thick, treacherous forest---called Riverwood by the Men---that had claimed many ships.  And so, while they could engage in water trade downstream, no trader came to Kobarthrond except by road and through the main gate.

``The days were peaceful and the nights were mild.  All a man can ask for", he replied.

Emptying his drink, Tarn put his mug down and wiped his beard with the back of his hand.  Lawrence had no beard at all to match his dark-red hair, though Tarn understood facial hair to be less common among Men.  Dwarves, on the other hand, grew their beards long by convention, braiding and decorating them with care.  Seeing anyone clean shaven, even a Man, and even a familiar Man like Lawrence, still felt odd.

``Can I buy you another?'' Tarn asked.  Then, jokingly, ``or would you prefer water?"

``Just because I don't bathe in the stuff like you do, doesn't mean I can't hold my own!"

Dwarves drank beer almost as much as they did water, and on social occasions like this there was no excuse to drink anything else.

``Anyway, I already tried to get some.  The bartender said they were out."

``Oh?''

``And not just today.  He said they'd been having trouble for weeks."

``Odd", said Tarn, trying to remember the last time he'd replenished his own water barrel.  The city had one primary well near the centre, going deep into the aquifer.  Other, smaller wells were connected to it.  Tarn had never known any of the wells to go dry.

``In fact", continued Lawrence", that's exactly what brings me to the city this time.  The king ordered a shipment of water from the river, and I just carted in eight full barrels of it."

Tarn, felt some level of responsibility for the city.  If there was a problem with Kobarthron's water supply, Tarn wanted to know about it.  Not that he could do much about it, but he felt some level of responsibility over the city---perhaps an inclination that came with his position as a guard---and wanted to keep on top of issues like this.  So he resolved to visit King Athzad and find out what was going on.

\chapter{Water}
Athzad, son of Valkold, was the king of Kobarthrond.