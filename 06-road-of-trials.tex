

Tarn and Peter found a nearby stream, had a drink and replenished their water skins.  Then Tarn asked, ``would it make sense to follow this downstream?  It might lead us to the coast.''

``Or to the swamp itself.  Swamps are usually found at the mouths of rivers.  I'm no scout or navigator, but that's probably the best chance we have of going the right way.''

Thus, the pair spent the morning following the stream.  Tarn tried to instruct Peter in how to use the dagger, while the cleric half listened and half hoped that he would never need to touch the weapon again.  The stream gradually got wider and deeper, until suddenly Peter stopped, pointing at the water.

``Look at that!'' he exclaimed.

``What is it?''

``The surface of the water.  Look at it!''

Tarn looked at the stream.  The water was behaving strangely, unnaturally.  It was swirling around, sometimes even flowing the wrong way.  The surface rose and fell, the stream swelling for a moment, then subsiding.  At the banks, too, it defied gravity: water seemed to trickle \emph{out} of the stream, up the banks and away from the river.  It seemed almost as if the water were alive, or being influenced by some magic.  They cautiously stepped closer.

Peter reached out with his walking stick, and touched the surface of the water.  The stick seemed unaffected, and the water continued its odd behaviour.  He then knelt down and stretched out his hand.

``Are you sure?'' Tarn whispered anxiously.

``Just a quick touch.''  He slowly lowered his hand, getting closer.  He lightly brushed the water with his fingers, then immediately withdrew his hand, reflexively, as if in pain.

Peter clutched his fingers while Tarn rushed to his side.  ``Are you okay?''

``Yes.  It didn't hurt; it was just \ldots\ cold.  Freezing, like a river through snow.''

Tarn glanced back at the stream.  ``Look!'' he yelled, pointing.

The water trickling up the banks suddenly slowed, and reversed direction.  The swirling stopped altogether.  The depth stayed constant.  The stream had suddenly started behaving normally again, as if in response to Peter's touching it.  Tarn looked at Peter puzzlingly.  The man shook his head slowly, dumbstruck.  Neither had any explanation for what they had seen.

Confused, they continued walking along, following the stream.  Any time they needed a drink, they were hesitant about touching or drinking the water, but they were able to do so without incident, and the odd behaviour never returned.  In time it grew into a wide river.

\chapter{Treeville}

By now it was afternoon, and Tarn suggested they find some food for supper.

``My bow was taken; how shall we hunt?'' Peter asked.

``I can use the goblin dagger.  Meanwhile, you could gather some firewood.''

There weren't many creatures around.  Tarn found the odd rabbit, and tried throwing the dagger, missing each time.  Sighing, he walked to pick it up, and continued searching.  He'd get used to the weapon eventually, he supposed.  He had to.

Suddenly, he heard a yell.  ``Tarn!''

The dwarf raced towards Peter's voice, grabbing his shield from his back and tightening his grip on the dagger.  He found him, alone and safe, but staring down at something on the ground: a body.

``Is it dead?'' Peter asked.  Tarn got closer to investigate. It was the corpse of a dwarf.  
He wore tattered leather armour, with a large dark patch on the chest.  Upon closer inspection it was a burn mark, going straight through the armour and into the skin.  The dwarf had met some fiery blow, and that may very well have been what killed him.  His boots were also old and worn.  Over his head was a green cowl made of fine cloth, soft to the touch and shimmering in the fading afternoon light.  Tarn removed it, and handed it to Peter.

``It feels a bit odd taking clothing from a dead person,'' Peter said as he reluctantly accepted the cowl.

``He has no use for it.  And besides, he died only recently.  There's no rot to worry about.  In fact,'' Tarn trailed off as he felt the burn mark, then mumbled ``still warm  \ldots''

Tarn stood up quickly, shield in hand, just as a figure emerged from the trees.

A female voice cried out; Tarn didn't recognise the words.  She was tall and lean, moving deliberately and gracefully as she stepped left, never taking her eyes off Tarn.  He thought she could have been an elf, if not for her skin: it was dark, almost black, with bright white hair.  He'd never seen a person that dark before.  She wore a crimson robe and cowl, and had a long stick in her hand.

Before Tarn could speak, the end of her stick became bright, glowing with light, and appeared distorted as if pulsing with great heat.  Then a sphere of fire materialised there, as if being drawn from the surrounding air.  She was clearly some kind of sorceress.  The fireball suddenly shot out towards Tarn at great speed, giving him barely enough time to raise his shield.  The fire crashed against it, splashing out around the edges and heating the metal until his arm was painfully hot, but the worst of it was blocked.

``Wait!'' shouted Tarn.  ``I am not your enemy.  I don't know this dwarf!''

The sorceress, clearly about to launch another attack, hesitated.  Tarn was too far away from her to pose an immediate threat with the dagger he held, and Peter was cowering near the corpse harmlessly.

``Drop your weapon,'' she commanded.  She spoke Human, but a broken dialect that made it very difficult for Tarn to understand. It was clearly not a native tongue to her, and Tarn himself had only a functional knowledge of the language.

If Tarn were disarmed, he would have no hope of fighting back.  But even if he held onto the dagger, he may not survive another fireball before getting close enough to attack.  He dropped it.

The sorceress asked a question.  Peter, breaking his silence, repeated the question for Tarn, translating it into clear, simple words: ``who are you?''

``I am a guard of Orehome, in the mountains to the north.  This man and I are alone in this country---we have no colleagues or collaborators here.  I don't know this dwarf or his allegiance.''  Peter repeated this for her, again using simpler and clearer words.

Tarn continued, as Peter translated.  ``My home city is in trouble, and I am seeking out the dark elves of Swampville, to request their help.''

She stared at Tarn for what felt like an hour, then said, ``put down your shield, I shall put away my staff, and we can talk.''  They did so.

The sorceress spoke, maintaining her aggressive tone.  ``I am Circe, a pyromancer; I am one of the Dark Elves of which you spoke.''  Tarn's eyes widened.  ``My people were from Swampville, west of here.  A hundred years ago we were raided by dwarves, a band of outlaws calling themselves the Exiles.  They razed our village and plundered everything we had.''

Tarn opened his mouth to offer his sympathy, but didn't know what to say.

``I am willing to entertain the possibility that you are not one of them,'' she continued, ``but I will take you to see my kin.  If they decide that you are not who you say you are, you will be killed.''

Peter and Tarn looked at each other.  ``It doesn't look like we have any choice in the matter,'' Tarn said, shrugging.  They allowed Circe to bind their hands, and she led them through the woods.

``If Swampville was razed, may I ask where we are going?'' Tarn asked.

``Very few of us survived the initial attack.  Since then we have taken refuge with our kin, the wood elves of Treeville.  It is their lord, Elrond, to whom I now deliver you.

``It is my turn to ask a question.  In what way do you expect the elves to be able to help a far-away Dwarven city?''

``Our water supply is corrupted.  I have heard a legend about a sword, teal in colour, that can slay the demon that may be responsible.  It is called `K\=\i{}ldir' and I was told it was in the possession of the dark elves of Swampville.''

The elf stopped, and looked at Tarn sympathetically.  ``I have heard of the sword you describe.  It was indeed in Swampville.  When the Exiles razed the village, they looted the sword.  To my knowledge, it is still in their possession.''  She continued walking.  They were silent until they reached Treeville, just as the sun was setting.

At first glance it seemed like an ordinary forest.  But as the sky darkened, and the trees grew denser, Tarn began to notice lights in the canopy.  Faint at first, then growing in number and brightness until the tops of all the trees seemed filled with tiny fires.

``You live in the trees?'' Tarn asked looking around, mouth agape with wonder.

``The Wood Elves do, and we follow their customs as their guests.''

As they went deeper into the forest, it became more and more evident that there was indeed a village hidden there.  The forest floor was peppered with gardens, small and odd-shaped so as to fit between the trees.  Tarn could see rings of vegetable sprouts and flax, and patches of flowers and mushrooms.  Small clusters of new trees, thin and woody, were also scattered around.

There were elves, too, tending to the gardens.  They didn't look like Circe, but more like Bookie back in Silverdale: pale skin, and yellow or brown hair.  Tarn guessed that these must be the Wood Elves.  They wore hooded cloaks and boots in browns and greens, making it hard to distinguish their figures from the trees and gardens.  They seemed almost to fade in and out of existence as they moved around, working.

The dark elf and her prisoners walked between the gardens, towards an especially wide tree trunk, with a dot of light at its base.  As they got closer, Tarn could see that it was a lantern, beautifully made of glass, with a large mushroom inside it giving off a beautiful warm glow.  Beside the lantern was a cavity in the tree trunk, into which Circe led them.  But it wasn't a hole or a tunnel; it was a staircase, cut into the trunk and winding around the outside.  More lanterns hung intermittently so that they could see where they were going.

``I've never seen anything like this,'' Tarn said.

``Nor I'', agreed Peter reverently.  Circe continued onwards.

After quite a bit of climbing, all the way around the trunk at least twice, they emerged onto a thick tree limb, large enough for them to walk comfortably between railings of sticks and vines that had been attached for safety.  At the end of the limb, where it spread out into countless branches, on top of which was a structure.  It looked to Tarn like a small shrine: tall, round, symmetrical, not big enough to be a house.  It was made of the very branches of the tree, winding up from the bottom to form the walls and roof.  Where the branches sprouted into leaves or flowers, the walls became patterned and beautiful.  It was very strange to Tarn: there was a tremendous amount of care and effort here, with the clear influence of a guiding hand, but at the same time it was constructed not by dominating the tree, but by working within it.  Part of him admired it, and part of him longed for it to have been built more \ldots\ explicitly, directly.  He felt like his soul was confused by it.

Tarn and Peter were led into the structure.  Inside was a long table surrounded by chairs, made from beautifully carved wood.  At the end of the table sat an older wood elf, wearing a very ornate robe of brilliant white, embroidered with thread that shone like the moon.  He rose, and said ``Greetings.  I am Elrond, son of Oldrond, lord of Treeville.  I apologise that I do not speak Dwarven.''  He had a strong command of the Human language, so that even Tarn could clearly understand him without much trouble.

``That's quite alright'', Tarn replied, before completing the introductions.  ``My name is Tarn, son of Rold, a son of Orehome in the northern mountain range.  My friend is Peter, a cleric of the Light from Silverdale on the Orehome River.''

Elrond then turned to Circe. ``What brings these guests to our forest?''

``My lord, I was patrolling the river east of here, and caught one of the Exiles.  After circling around, I found his body being inspected by this dwarf and this man.  The man seems harmless, and the dwarf claims to be unassociated with the Exiles and to have peaceful intentions.  He said he intended to travel to Swampville, seeking aid.''

``My good dwarf, it may trouble you to know that Swampville is no more.  It was destroyed by this rogue band of dwarves, these `Exiles'.

``Circe told me'', replied Tarn.  ``It saddened me to hear it.  I cannot imagine the pain of losing one's home \ldots\ even if it was generations ago.''

Elrond smiled.  ``A generation is not as sensible a measure of time, or of experience, for Elves as it is for the mortal races.  We do not perish save from bloodshed or serious disease, and so whereas your society and culture ebb and flow as each passing wave of elders give way to those coming of age, ours do not.  We have a continuity that transcends centuries.  In addition to this, we are not as proliferant, and so there is no calculable period of time that could even reasonably be called a `generation'.  There is no standard age at which elven women tend to bear their first child, as there is for human or dwarven women.  Most have no children at all.

``Circe is a young elf, but she has already lived longer than a whole lifetime of a Man or a Dwarf.  And so, even though Swampville was looted and razed nearly one hundred years ago, Circe was herself there; she is a survivor of that atrocity.  She did indeed lose her home, and continues to carry the grief with her.  I am sure she appreciated your words.''

Tarn looked at Circe.  The young pyromancer was staring into the distance, as if reliving painful memories.

Lord Elrond continued.  ``The settlement may be gone, but the colony persists, its survivors and their descendants living in refuge here in Treeville.  That leads to my next question, Tarn son of Rolg: what aid do you seek from the dark elves?''

``I explained this to Circe earlier.  My city has a corruption in our water supply, and I seek the legendary sword K\=\i{}ldir which has been known to eliminate similar pollution in the past.  I understand that the sword has been taken by the Exiles.'' Then, supposing it may help his case if he were to demonstrate that he was not an enemy of elves, he added ``The librarian in Silverdale---Bookie, an elf---shared the legend with me''

``I assume you are referring to the song, \emph{The Land of Water}?''
Tarn nodded.
``I know it well.  Bookie is one of our kin, and that was one of many tomes and scrolls that he took with him when he left Treeville to establish his library in Silverdale.  And you are correct: K\=\i{}ldir was indeed taken by the dwarves in their raid on Swampville.``

Elrond then turned to Peter.  ``And what about you, cleric Peter?  What brings you so far from Silverdale?''

``I was traveling with Tarn here towards Westport, when we were attacked by goblins.  We lost our horses and money.  I have no means of completing my journey, and I have come to see Tarn's quest as worthy of assistance.  So I decided to forget about Westport for now, and instead help my new friend find the dark elves and the sword.''

Elrond looked attentively at Peter while the man spoke, and after a moment declared, ``I choose to believe your story---both of you---and so I will help you to whatever extent I can.  We will provide you with rest and accommodation until you depart our hospitality, and then with provisions.  The best assistance I can offer, though, is information---we here in the forest have taken up the burden of recording the histories and knowledge of the region.  You are fortunate to find yourself in the best possible place to plan your next steps.  So please ask any questions that you feel may be relevant, and I will do my best to answer them.''

There was only one issue at the forefront of Tarn's mind, and until that was resolved, everything else was clouded.  So he asked, ``do you know where we can find the sword now?''

Elrond smiled.  ``I will answer your question.  But first, know that it rests on a premise that may or may not be sound.

``The sword K\=\i{}ldir was forged long ago, in a Dwarven City floating above the sea.  It is unknown how it was constructed or kept above the waves.  Just as far back in the mists of antiquity, a calamity befell the city and it was destroyed.  Treeville did not exist yet; there were no elves or other lorekeepers in the region, and thus no records now exist of how it was constructed or kept above the waves, nor of how it fell.  All documents and almost all artefacts belonging to the city itself were destroyed by the calamity.  All we have is the legend.  We now call the place V\=ald\=unmir, \emph{the sunken city}, though its actual location is, like all else, a mystery.

As you know, according to the legend, K\=\i{}ldir was used to vanquish the corruption in the city's drinking water.  But just like the city itself, we do not know by what magic or mechanism the sword worked.  We do not know how to use it.  To our knowledge no elf does, nor man, nor dwarf, including the villains who took it while raiding Swampville.  The knowledge is forgotten.  It may involve some sorcery from a lost school of magic; it may require an incantation in a dialect that died with the city; the practitioner may need to perform an esoteric ritual; the vanquishing of a demon---if indeed we are to take that part literally---may be achieved by commune with some power that is higher still.  That legend of the Land of Sea, passed down over the centuries, is the only suggestion of fact that we have, and all it tells us on this matter is that it `pierced the water'.

``Therefore, even if you manage to recover the sword, there is no guarantee that you can actually use it to help make Orehome's water drinkable once more.  Even putting aside the mystery surrounding the sword's use, Orehome's water may simply suffer from a different kind of corruption, one which K\=\i{}ldir is powerless to address.  I hope that you understand this.

``I do,'' Tarn said in a serious tone, nodding.  What Elrond just mentioned had occurred to Tarn: the sword may indeed not help with Korbarthrond's specific problem.  But he hadn't really given it much thought.  His focus was still, immaturely perhaps, on acquiring the sword itself, the renown and importance of the weapon, and the special nature of its craft.  He of course kept his ultimate goal in mind, but it was easy to dismiss concerns like this when they were not the focus of his attention.  \emph{It will probably work}, he would tell himself.  \emph{It must work}.  Then he would go back to imagining finding, and holding, and keeping the sword, and that concern would be forgotten once again.

``Thousands of years ago,'' said Elrond, ``the dark elves of Swampville discovered the sword while diving off the coast near their village.  They had heard the legend, but until the sword itself was found it was supposed that V\=ald\=unmir may not have ever really existed.

``Over time, they recovered a small collection of other objects from V\=ald\=unmir.  I believe that when the Exiles destroyed Swampville, they were driven at least in part by a desire to burglarise these dwarven artefacts.  Or perhaps, as they see it, to liberate them.  There had historically been some animosity between the dwarves and the elves of the region, and it would not have been easy to see parts of their culture and history in the hands of adversaries.''

``But we took every measure to \ldots'' interjected Circe defensively.

Elrond raised his hand and she stopped speaking.  ``I know that Swampville acted as good-faith lorekeepers, preserving and documenting the history of V\=ald\=unmir.  I merely suggested that the dwarves may not have seen it that way.  It is unfortunate that their interpretation led to actual conflict, but that is where we currently stand.

``The Exiles occupy the hills south of here.  They do not travel this far north very often, but when we come into contact it is generally violent.  I hear that you met Circe here under such circumstances.  Whether you wish to go in search of these dwarves, or will return home with this new information, is up to you.  Either way, our hospitality is yours and, as I have previously said, we will do what we can to help you prepare for your journey.''

Peter spoke up.  ``Excuse me, but I have another question.''

``Of course.  How may I help?''

``Earlier today, while we walked along the river, we saw some strange behaviour in the water at one place.  It flowed in strange directions, rose and fell rapidly, and was freezing cold to the touch.  Do you know what might have caused that?''

Without hesitating, Elrond answered, ``it sounds like you came across a frost elemental.  They are sentient, and exert influence over the waters of the world.  They are generally invisible but their effects can sometimes be observed.''

Peter was dumbfounded.  ``It was \ldots\ alive?''

``Quite so.  But the nature of Elementals is not well understood.  Communicating with one is even rarer than seeing one.  Who can say how large one might be, or how far its influence can extend?  Or, more importantly, what its purpose might be?  The Elementals are one of many great mysteries in our world, and where it doesn't endanger you, the most satisfying mindset is to appreciate such a mystery that does not want to be solved.  A scholar could camp by a river for a lifetime and never observe one, while you who happened to be walking nearby saw one, saw its influence, and even touched it.  If I were you I would appreciate the rare experience, but not dwell too much on its meaning.  You may never comprehend it.''

``Thank you for explaining,'' Peter said.  Then, ``oh, one more question, if you will.''

``Yes?'' Elrond asked pleasantly.

Peter pulled the green cowl from his pocket and presented it.  ``This cowl was recovered from the dwarf killed today by Circe.  It feels odd to the touch; it's hard to describe.  Like there's an aura coming off it.  Can you tell me anything about it?''

``I do not know much about the dwarven way of making things, but I will tell you what I can.''  He took the cowl from Peter.

After inspecting it for a short time, he spoke.  ``It was spun from sheeps' wool, and dyed green with forest herbs.  There is an enchantment placed upon it, which slightly improves the wearer's focus and concentration.''  He returned the cowl to Peter.

``Wow, thank you!'' Peter said.

``It is my pleasure.  Now, is there anything more I can do to assist you this evening?''  After a pause, he continued.  ``In that case, I suggest that you get some sleep.  You have had a long and eventful day.''

``Follow me to your quarters, please,'' a guard said politely.  He wore the same uniform Tarn had seen on the other guards posted around the village: a handsome outfit, mostly dark fabric and leather in brown and green, but with pauldrons, a breastplate and a tall helmet made of metal.  These pieces were beautifully polished, and the metal had the shine and colour of honey---a copper alloy, Tarn guessed, like a bronze or a brass.  He carried a long spear, the head of which was made from that same bright metal.  Tarn couldn't precisely identify the metal, and so he couldn't tell whether the armour and weapon were more ceremonial or functional.

The guard led them back down to the forest floor, then towards a much smaller tree.  It was short enough that Tarn could see the top of it.  Nestled within the boughs was a small cabin, of similar construction to Elrond's meeting hall.  As the only apprent way to get uyp or down this tree, the cabin had a rope ladder hanging down.  The guard told them to climb; that this was their guest house for as long as they stayed in the forest.

They climbed the ladder up to a small balcony outside the cabin, and walked inside the door.  They found two carved wooden beds, with linen pillows and blankets that looked thick and warm.  There was also a large jug of water with a basin, a table and chairs of wood, and a cupboard filled with various vegetables and herbs.  Once they entered the cabin they realised just how exhausted they were, and both Tarn and Peter immediately climbed into the soft beds, quickly falling asleep.

\chapter{The pyromancer}

``Tarn?  Peter?''  Circe was knocking on the cabin door, calling their names.  It was late morning.  The travelers climbed out of their beds, quickly washed and dressed, and opened the door.  ``Good morning!  I hope you found the beds agreeable.''

``I for one certainly did,'' replied Tarn cheerfully, ''but having spent the previous night in a pile of mud, I'm sure I was easy to please.''

``Elrond has discussed your matter with the council, and has decided that if you were to seek out the Exiles, Treeville will not get involved.  Your quest is considered unimportant.''

Tarn scowled, insulted, but Circe went on, ``please don't take it personally.  Your city's water supply is of course important to \emph{you}, but we Elves prefer focusing on larger affairs---long timescales.  Or on the work we have taken on as a duty to the world, like growing plants, weaving, creating medicines, and preserving history.  Or on maintaining balance.  The council have deemed that your problem is not a matter of balance, and they overlook even an entire town's destruction as being worthy of revenge.''

``Well, I didn't really expect them to send an army with me.  Come to think of it, we haven't even spoken about what we plan to do,'' Tarn said, turning to Peter.  In truth, the only course of action that Tarn had even considered was to go after the sword.  ``But I do intend to look for these Exiles and try to recover the sword.''

``And I, of course, will accompany you, as I have promised to do.'' said Peter.

Circe spoke: ``should you find the sword, the elves of Treeville and Swampville will allow you to take possession of it and return with it to your mountain home.''

``That's a relief'', Tarn said, ``and very generous of the people from whom it was stolen.  Please convey my assurance to your people that I am sincere in my quest, and desire only to help my people.''  At those words he felt a bite of guilt.  He was again using Korbarthrond as an excuse, while his true purpose was merely in acquiring the legendary sword.  \emph{Nevertheless}, he thought, \emph{I do indeed plan to help the city}.  He tried to maintian a neutral expression.

``It was stolen from us, yes, but the weapon was dwarf-made before it ever came into our hands.'' answered Circe.  Then, she smiled.  ``That was my acting as an envoy for my people.  I now speak for myself.  Tarn, son of Rolg, if you go to fight the villainous dwarves, then I ask that you allow me to accompany you.''

``Accompany me?  I thought my task was `unimportant'.''

``Unimportant to the elves as a people, yes; but not to me.  I believe that the Exiles are up to something sinister, which is not a view shared by my kin.  I see it as my duty to put this right.''

Tarn suspected that she, too, had a concealed motivation: revenge for the home she lost at the hands of the Exiles.  He kept this suspicion to himself, though, for the sake of politeness, and because there was nothing to be gained from antagonising her.

``Your offer to join us is very generous, but \ldots'' Tarn trailed off, looked hesitant.

``I am a skilled pyromancer, as you have seen first hand.  I am sure that I would be useful.''

``I don't doubt your ability,'' he said reassuringly.  ``It's just that \ldots\ well, I don't know if a fight will even be necessary.  What if we meet the Exiles, explain that Orehome is in need, and they give up the sword willingly?''

Circe looked incredulous.  ``I've fought with these scoundrels for years.  They do not have good hearts, and I know that they will not give up their treasure to help an outsider.''

``I don't want the blood of innocents on my hands, nor of any person at all if it can be avoided.  You are also assuming that we would not be hurt or killed in a violent confrontation.''

``I am sure that we would win the day,'' she said simply.

Tarn paused for a moment, then spoke as authoritatively as he could.  ``We would welcome your company on the road, and your counsel, and your help if required.''

``Thank you!'' she said, but Tarn cut her off.

``However: we will use violence only as a last resort.  Diplomacy should take preference.  If that fails---if an attack is the only method left to us---then, and only then, shall we employ it.  Is this an acceptable compromise?''

Circe considered, then said ``Yes.''

``Welcome to the team!'' said Peter, holding out his hand and shaking hers.