\chapter{Treeville}

Tarn and Peter found a nearby stream, had a drink and replenished their water skins.  Then Tarn asked, ``would it make sense to follow this downstream?  It might lead us to the coast.''

``Or to the swamp itself.  Swamps are usually found at the mouths of rivers.  I'm no scout or navigator, but that's probably the best chance we have of going the right way.''

Thus, the pair spent the morning following the stream.  Tarn tried to instruct Peter in how to use the dagger, while the cleric half listened and half hoped that he would never need to touch the weapon again.  The stream gradually got wider and deeper, until suddenly Peter stopped, pointing at the water.

``Look at that!'' he exclaimed.

``What is it?''

``The surface of the water.  Look at it!''

Tarn looked at the stream.  The water was behaving strangely, unnaturally.  It was swirling around, sometimes even flowing the wrong way.  The surface rose and fell, the stream swelling for a moment, then subsiding.  At the banks, too, it defied gravity: water seemed to trickle \emph{out} of the stream, up the banks and away from the river.  It seemed almost as if the water were alive, or being influenced by some magic.  They cautiously stepped closer.

Peter reached out with his walking stick, and touched the surface of the water.  The stick seemed unaffected, and the water continued its odd behaviour.  He then knelt down and stretched out his hand.

``Are you sure?'' Tarn whispered anxiously.

``Just a quick touch.''  He slowly lowered his hand, getting closer.  He lightly brushed the water with his fingers, then immediately withdrew his hand, reflexively, as if in pain.

Peter clutched his fingers while Tarn rushed to his side.  ``Are you okay?''

``Yes.  It didn't hurt; it was just \ldots\ cold.  Freezing, like a river through snow.''

Tarn glanced back at the stream.  ``Look!'' he yelled, pointing.

The water trickling up the banks suddenly slowed, and reversed direction.  The swirling stopped altogether.  The depth stayed constant.  The stream had suddenly started behaving normally again, as if in response to Peter's touching it.  Tarn looked at Peter puzzlingly.  The man shook his head slowly, dumbstruck.  Neither had any explanation for what they had seen.

Confused, they continued walking along, following the stream.  Any time they needed a drink, they were hesitant about touching or drinking the water, but they were able to do so without incident, and the odd behaviour never returned.  In time it grew into a wide river.

By now it was afternoon, and Tarn suggested they find some food for supper.

``My bow was taken; how shall we hunt?'' Peter asked.

``I can use the goblin dagger.  Meanwhile, you could gather some firewood.''

There weren't many creatures around.  Tarn found the odd rabbit, and tried throwing the dagger, missing each time.  Sighing, he walked to pick it up, and continued searching.  He'd get used to the weapon eventually, he supposed.  He had to.

Suddenly, he heard a yell.  ``Tarn!''

The dwarf raced towards Peter's voice, grabbing his shield from his back and tightening his grip on the dagger.  He found him, alone and safe, but staring down at something on the ground: a body.

``Is it dead?'' Peter asked.  Tarn got closer to investigate. It was the corpse of a dwarf.  
He wore tattered leather armour, with a large dark patch on the chest.  Upon closer inspection it was a burn mark, going straight through the armour and into the skin.  The dwarf had met some fiery blow, and that may very well have been what killed him.  His boots were also old and worn.  Over his head was a green cowl made of fine cloth, soft to the touch and shimmering in the fading afternoon light.  Tarn removed it, and handed it to Peter.

``It feels a bit odd taking clothing from a dead person,'' Peter said as he reluctantly accepted the cowl.

``He has no use for it.  And besides, he died only recently.  There's no rot to worry about.  In fact,'' Tarn trailed off as he felt the burn mark, then mumbled ``still warm  \ldots''

Tarn stood up quickly, shield in hand, just as a figure emerged from the trees.

A female voice cried out; Tarn didn't recognise the words.  She was tall and lean, moving deliberately and gracefully as she stepped left, never taking her eyes off Tarn.  He thought she could have been an elf, if not for her skin: it was dark, almost black, with bright white hair.  He'd never seen a person that dark before.  She wore a crimson robe and cowl, and had a long stick in her hand.

Before Tarn could speak, the end of her stick became bright, glowing with light, and appeared distorted as if pulsing with great heat.  Then a sphere of fire materialised there, as if being drawn from the surrounding air.  She was clearly some kind of sorceress.  The fireball suddenly shot out towards Tarn at great speed, giving him barely enough time to raise his shield.  The fire crashed against it, splashing out around the edges and heating the metal until his arm was painfully hot, but the worst of it was blocked.

``Wait!'' shouted Tarn.  ``I am not your enemy.  I don't know this dwarf!''

The sorceress, clearly about to launch another attack, hesitated.  Tarn was too far away from her to pose an immediate threat with the dagger he held, and Peter was cowering near the corpse harmlessly.

``Drop your weapon,'' she commanded.  She spoke Human, but a broken dialect that made it very difficult for Tarn to understand, he himself having only a functional knowledge of the language.

If Tarn were disarmed, he would have no hope of fighting back.  But even if he held onto the dagger, he may not survive another fireball before getting close enough to attack.  He dropped it.

The sorceress asked a question.  Peter, breaking his silence, repeated the question for Tarn, translating it into clear, simple words: ``who are you?''

``I am a guard of Orehome, in the mountains to the north.  This man and I are alone in this country---we have no colleagues or collaborators here.  I don't know this dwarf or his allegiance.''  Peter repeated this for her, again using simpler and clearer words.

Tarn continued, as Peter translated.  ``My home city is in trouble, and I am seeking out the dark elves of Swampville, to request their help.''

She stared at Tarn for what felt like an hour, then said, ``put down your shield, I shall put away my staff, and we can talk.''  They did so.

The sorceress spoke, maintaining her aggressive tone.  ``I am Circe, a pyromancer; I am one of the Dark Elves of which you spoke.''  Tarn's eyes widened.  ``My people were from Swampville, far west of here.  A hundred years ago we were raided by dwarves, a band of outlaws calling themselves the Exiles.  They razed our village and plundered everything we had.''

Tarn opened his mouth to offer his sympathy, but didn't know what to say.

``I am willing to entertain the possibility that you are not one of them,'' she continued, ``but I will take you to see my kin.  If they decide that you are not who you say you are, you will be killed.''

Peter and Tarn looked at each other.  ``It doesn't look like we have any choice in the matter,'' Tarn said, shrugging.  They allowed Circe to bind their hands, and she led them through the woods.

``If Swampville was razed, may I ask where we are going?'' Tarn asked.

``Very few of us survived the initial attack.  Since then we have taken refuge with our kin, the wood elves of Treeville.  It is their elder, Elrond, to whom I now deliver you.

``It is my turn to ask a question.  In what way do you expect the elves to be able to help a far-away Dwarven city?''

``Our water supply is corrupted.  I have heard a legend about a sword, teal in colour, that can slay the demon that may be responsible.  It is called `K\=\i{}ldir' and I was told it was in the possession of the dark elves of Swampville.''

The elf stopped, and looked at Tarn sympathetically.  ``I have heard of the sword you describe.  It was indeed in Swampville.  When the Exiles razed the village, they looted the sword.  To my knowledge, it is still in their possession.''  She continued walking.  They were silent until they reached Treeville, just as the sun was setting.

At first glance it seemed like an ordinary forest.  But as the sky darkened, and the trees grew denser, Tarn began to notice lights in the canopy.  Faint at first, then growing in number and brightness until the tops of all the trees seemed filled with tiny fires.

``You live in the trees?'' Tarn asked looking around, mouth agape with wonder.

``The Wood Elves do, and we follow their customs as their guests.''

As they went deeper into the forest, it became more and more evident that there was indeed a village hidden there.  The forest floor was peppered with gardens, small and odd-shaped so as to fit between the trees.  Tarn could see rings of vegetable sprouts and flax, and patches of flowers and mushrooms.  Small clusters of new trees, thin and woody, were also scattered around.

There were elves, too, tending to the gardens.  They didn't look like Circe, but more like Bookie back in Silverdale: pale skin, and yellow or brown hair.  Tarn guessed that these must be the Wood Elves.  They wore hooded cloaks and boots in browns and greens, making it hard to distinguish their figures from the trees and gardens.  They seemed almost to fade in and out of existence as they moved around, working.

The dark elf and her prisoners walked between the gardens, towards an especially wide tree trunk, with a dot of light at its base.  As they got closer, Tarn could see that it was a lantern, beautifully made of glass, with a large mushroom inside it giving off a beautiful warm glow.  Beside the lantern was a cavity in the tree trunk, into which Circe led them.  But it wasn't a hole or a tunnel; it was a staircase, cut into the trunk and winding around the outside.  More lanterns hung intermittently so that they could see where they were going.

``I've never seen anything like this,'' Tarn said.

``Nor I'', agreed Peter.  Circe continued onwards.

After quite a bit of climbing, all the way around the trunk at least twice, they emerged onto a thick tree limb, large enough for them to walk comfortably between railings of sticks and vines that had been attached for safety.  At the end of the limb, where it spread out into countless branches, on top of which was a structure.  It looked to Tarn like a small shrine: tall, round, symmetrical, not big enough to be a house.  It was made of the very branches of the tree, winding up from the bottom to form the walls and roof.  Where the branches sprouted into leaves or flowers, the walls became patterned and beautiful.  It was very strange to Tarn: there was a tremendous amount of care and effort here, with the clear influence of a guiding hand, but at the same time it was constructed not by dominating the tree, but by working within it.  Part of him admired it, and part of him longed for it to have been built more \ldots\ explicitly, directly.  He felt like his soul was confused by it.

Tarn and Peter were led into the structure.  Inside was a large table surrounded by chairs, made from beautifully carved wood.  At the end of the table sat an older wood elf, wearing a very ornate robe of brilliant white, embroidered with thread that shone like the moon.  He rose, and said ``Greetings.  I am Elrond, Elder of Treeville.''  Then, turning to Circe, ``what brings these guests to our forest?''


Some metal?