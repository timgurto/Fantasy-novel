\chapter{The Library of Silverdale}

When they reached the city, Tarn and Lawrence went their separate ways.  Tarn put his shield on his back and his hammer in his belt, holding his helmet under his arm.  He was happy to leave the remaining beer with Lawrence, who directed Tarn to the library before leaving for the town markets.

Silverhome was a medium-sized town of men.  Fewer souls than \korbarthrond, he thought, but it seemed to cover a wider area---and that wasn't even counting the widespread farms outside the town proper.  Tarn marveled at the human buildings: all free-standing, and constructed from wood, stone metal; whatever was most functional.  And they were tall: some two storeys high!  In \korbarthrond\ everything significant was carved into the rock, with the only free-standing structures being small things like tents or market stalls.  The technical achievement boasted by these human buildings was impressive, but they also had a consistent failing: they were not beautiful.  Fit for purpose and well-built, certainly, but the builders clearly focused on utility and left symmetry, finishing and decoration by the wayside.  A cultural difference, Tarn supposed, which he must simply accept.

After the buildings, the next thing that caught Tarn's eye was all of the animals.  A shepherd walked along the road leading three large sheep. A man with a bow and a knife, whom Tarn guessed was a hunter, strolled along with a fierce-looking dog at his heel.  A knight wearing an elaborate plumed helmet and polished steel armour rode past on a well-kept horse.  At home, Tarn could go weeks without seeing an animal; aside from those slaughtered for meat, the only other animals he knew of in \korbarthrond\ were the chickens kept for their eggs.

In addition to the men with their animals, Tarn did spot the occasional dwarf.  They were craftsmen, carrying special materials that could only be bought here, or trying to sell their wares.  They were exceptional though: most craftsmen stayed in \korbarthrond\ and waited for the merchants to come to them.

Following Lawrence's directions, Tarn found the library.  He knew enough of the Human tongue that he could understand the sign on the building, so he walked through the doorway.

Just inside the front door was a desk, staffed by an odd-looking person, the likes of which Tarn had never seen before.  He was tall, taller than most men, with skin pale and very smooth.  He had very long ears, and wore no beard.  Based on stories he had heard, Tarn could only assume that this creature was an elf.

Waving, the person said ``Gr\=urg, tu ski,''.

``Grurg,'' Tarn echoed hesitantly, returning the greeting but surprised to hear his native tongue.

``Do you speak Human?  I know enough Dwarven to be polite, but I have never really had occasion to learn or practice it.''

``Yes I do, well enough I suppose.''

``Excellent!  My name is Ithur, and I am the librarian here.  How can I help you?''

``Uhmm \ldots\ are you an elf?'' Tarn stammered, trying not to sound rude.

``Yes I am.  A wood elf, to be precise.  Am I the first you've seen of my kind?''

Tarn nodded.  The elf seemed to have a strong sense of purpose whenever speaking or moving; deliberate and slow, yet elegant and efficient.  It was strangely pleasant to listen to him speak and to watch him work.

``There is none other in Silverdale, and so I may also be the only elf that you ever see hence.''

After a moment of silence, Tarn came back to his senses.  ``Oh, err \ldots\ I've come looking for a solution to a problem befalling our city.  I was hoping the men of this town might have a solution that we do not.''

``Your city being Orehome?''  Tarn nodded.

``I may not be a man, but my position here is as a keeper of history, legend and truth.  In fact, I believe there is no other such record anywhere in the town.  Therefore, if there is a solution to be found in Silverdale, it may very well be in my library. If you would: what is the nature of Orehome's problem?''

Tarn told Ithur about the poisoned wells, the gut sickness, and that boiling the water is an apparent solution.  The librarian thought for a few moments, and then spoke.

``I have known of individual vessels of water becoming tainted, and the solutions I have seen are to boil the water---as you have said---or to discard it and fetch a fresh load.  I have not seen an entire well suffer from this that had previously been pure.

``That being said, I do recall an old legend about another town that was forced to import water due to some blight.  Please give me some time to try to find it.''

``Go right ahead,'' answered Tarn.

Ithur walked away from the desk, to shelves overflowing with books and scrolls.  They were stuffed into every available space, with no apparent system governing them, and yet Ithur seemed able to easily find anything that he was seeking.  It seemed strange that a creature exuding such discipline and control could be responsible for this mess.

After about fifteen minutes of searching, reading and cross-referencing, Ithur returned with a scroll.  ``This is the legend of which I spoke: \emph{the Land of Sea}.''  He unrolled it and read:

\settowidth{\versewidth}{Beyond the Western coast, above the waves,}
\begin{verse}[\versewidth]\poem{The Land of Sea}
Beyond the sandy coast, above the waves,\\*
by unknown arts the Land of Sea was grown,\\
for deep below, in ancient sunken caves\\*
enchanted metals waited in the stone.

The Land of Sea was rock made smooth and true.\\*
The architects a city founded there,\\
and from the coast a citizenry drew\\*
who came in ships inspired by the dare.

They delved beneath the waves with pick in hand,\\*
and metal ores and gemstones rare they won;\\
with ironwork they added to the land\\*
and food they grew, in fields beneath the sun.

The city thrived, but deep beneath the tides\\*
a dark corruption fouled the seas like ink.\\
And so, despite the waters on all sides,\\*
the Land of Sea had nothing fit to drink.

The merchant bartered gold for water plain;\\*
the chemist boiled the brine and saved the steam;\\
the plumber's pipe and barrel caught the rain;\\*
the metalworker bold began to scheme \ldots

A sword was made, with wizardry imbued;\\*
to slay the villain was its lofty aim,\\
pale teal in colour, mirror-like when viewed,\\*
and \kildir, \emph{Clarity}, its chosen name.

A hero faced the dark with sword in hand\\*
and pierced the water, thrusting deep and sure.\\
He smote the demon, watching it disband,\\*
and blessed the land with water clean and pure.
\end{verse}

``That was quite a story,'' Tarn remarked cynically.

``A fantastic myth, to be sure.  I cannot offer you any certainty that the Land of Sea ever really existed, or if it did, whether it truly floated above the waves; much of the story may be mere poetic flourish.  However, the sword itself is a known artefact.''

Tarn's eyes widened.  ``The sword is real?''

``It is,'' replied the elf.  ``\kildir\ exists, but whether it ever slew a water-corrupting demon is, like much of the myth, impossible to verify.''

Tarn realised something.  ``Hold on, if the sword's name is `\kildir', does that mean the Land of Sea was a city of dwarves?''

``It is impossible to say.  This myth was written in Elvish, but it does explicitly use the Dwarven script when naming the sword.

He showed Tarn the scroll and pointed to the lone Dwarven word among the sea of Elvish curves: \dwarven{\kildir}.

Ithur continued speaking.  ``The story may have been recounted to the author by a dwarf, or perhaps a Dwarf simply named the sword.  It is, again, impossible to say.  I suspect that the Dwarven name is engraved on it, and the author merely reflected that.  The weapon itself is currently being held by elves.''

Tarn was beside himself.  Not only was mythical sword real, but this librarian knew where it could be found!  It could even have been an ancient Dwarven relic.  A sword so impressive that it has a song written about it!  And it was within his grasp \ldots

He took a deep breath.  ``Where can I find these elves?''

``I would be happy to tell you, but remember that I cannot verify that the sword will actually help your city with its problem.''

``Huh?'' Tarn interjected, confused.  He had been so fixated on the sword itself, as a rare and legendary artefact, a prize to be acquired, that he had forgotten that his true purpose here was to solve \korbarthrond's water problem.  ``Oh, of course.  Well, even the possibility is worth exploring, don't you think?''

Ithur raised a suspicious eyebrow, and continued.  ``To my knowledge, \kildir\ is in the possession of a colony of Dark Elves, inhabiting a swamp on the Western coast.  The colony is called \yedmurdim.  If you wish to seek it out''---he paused, and observed Tarn nodding enthusiastically---``then the best way would be by ship from Westport, south along the coast.''

Tarn glanced at the door, eager to leave.  The only thing that occupied his mind right now was the mystical teal-coloured sword: what it must be like to see it, to hold it as that hero from the song held it.  To possess and to treasure.

``Before you go,'' said Ithur, deflating Tarn and bringing his thoughts back into the room, ``let me give you something for the journey.''  He rummaged around in his desk, and produced a small cloth pouch which he gave to Tarn.  ``Some healing herbs,'' he said.

``I thought you were a librarian, not an apothecary,'' Tarn said, smiling.

``I am, but I am also an elf.  Although I am alone in this town, I preserve my roots by maintaining certain traditional pasttimes.  I imagine if you were the only dwarf in your town that you would do something similar.''  Tarn had trouble imagining that scenario.

Ithur explained, ``I keep a herb garden at my home, and engage in some amateur alchemy.  They are historic and respected vocations of Elfkind, and I find them to be pleasant and comforting hobbies.  I also wove that pouch holding the herbs, and these clothes.''  He gestured at his robe, which was clearly made with an unskilled yet loving hand.  Tarn may not have known much about tailoring, but he could recognise the different facets of craftsmanship.

``Thank you for the gift,'' said Tarn.

``My pleasure!  If you are hurt and bleeding, rub the herbs onto the wound.  They will help to prevent an infection.''

``I will.  And thank you for sharing the myth of the Land of Sea.''

``All I ask in return,'' replied Ithur, ``is that if you learn anything on your journey, that you tell me so that I may fill in the hazier parts of these myths and histories.''

Tarn nodded, and stepped out of the library.


%\chapter{By Horse and Pony}
As the dirty street air drifted into his lungs and the clamouring of merchants and children and animals filled his ears, Tarn came back to his senses.  His intention had been to visit the library, find a solution if one existed, then return home.  A simple journey, quick, with no risks.  And now he felt he was being pushed into yet another adventure.  Or rather pulled: it was the allure of that sword that had enchanted him.  The desire for it.

It was as simple as that.  It outweighed the fear of travel, or the skepticism that a solution might be out there.  Of course, this may indeed turn out to be a solution.  That possibility made it easier for Tarn to justify this drive: he wasn't abandoning his city---he was still trying to save it!  In any case, his heart was already on the journey, and his feet could do nothing now except follow.

Tarn headed back towards the town's gate, and entered the stable there.  It was filled with horses for men and ponies for dwarves, who were too short to mount or dismount a full-sized horse.

``Good afternoon, my good dwarf,'' the proprietor said.  ``What can I do for you?''

``I'd like to hire a pony, to take me to Westport.''

``Good, good.  We have a branch in Westport, so once you arrive you can simply return the pony to the stables there.''  The man then gave Tarn a look up and down, and frowning, asked ``is this all you are taking?''

Tarn stumbled over his words.  He had rushed here without thinking about what he'd need, how long the journey would be, or any of the details.  ``Erm \ldots''

At this, another man approached Tarn from behind and tapped him on the shoulder.

``Good day.  I am also looking to travel to Westport; would you perhaps like to accompany me on the road?  My name is Peter.''

Tarn turned and looked up.  Peter had short blonde hair and a young, kind face.

``I can help you to organise your provisions, too,'' Peter offered.

``\ldots\ I would appreciate that,'' Tarn eventually answered.  He didn't know what to expect from the partnership, but at worst, he thought, he could split off from this stranger once they left the town.

Tarn hired his horse with some of the gold coins from his purse.  It was not a trivial amount of money, but the gold wasn't doing much good piling up in Tarn's quarters in the mountain.  It was not the loss of wealth that made him hesitate, but rather the loss of the coins themselves.  It was painful to part with treasured possessions, but when weighed against the possibility of his new prize, he could bring himself to make the transaction.

Peter and Tarn took their horses, and hitched them at an inn while they prepared for the journey.  It was mid-afternoon, and so they would need to leave soon if they were to get any distance behind them by nightfall.  They bought food, bedrolls, and a hatchet for firewood. They filled water skins.  Peter brought a walking stick, and a bow with some arrows for hunting, and Tarn had his hammer, shield and helmet in case they met any trouble on the road.

Everything packed, the pair mounted up and rode out through the town gate.  This side of the river there was only one road leading out of Silverdale, heading back to the mountain.  A few miles along it split, and the new fork led south, all the way around the entire mountain range before straightening out towards the western coast and Westport.

As Tarn got used to his new mode of transportation, Peter offered him a gift: a thin silver necklace, with a small medal hanging off it in the shape of the radiant sun.

``What's this?'' Tarn asked.

``A religious symbol.  I am a cleric of the Order of Light.  This necklace may help to protect you on our journey.''

Tarn was familiar with the Order.  Many dwarves were followers, and there were a handful of shrines and temples of the religion within \korbarthrond.  He himself was not a follower, and didn't give much thought to matters of the soul or the spirit world, but he didn't begrudge others their consolations.  The necklace itself, though, was a fine-looking thing.  It didn't contain much metal, but it was made with purpose and care, and it was offered as a gift.

``I'm not a believer, but I would be glad to wear this,'' he said, careful not to drop it as he stretched his hand out to accept.  Then, ``is it religious business that takes you to Westport?''

``Indeed it is!  I wish to attend the cathedral in the city, to learn from my superiors and to study in the library there.  What about you?  If you don't mind my asking.''

Tarn thought about this: how much should he say?  Might this cleric want to take the sword for himself?  Might he want to help Tarn to seek out the elves?  Might he himself know of a solution to the water problem?

Eventually he decided to tell Peter the simplest version of the truth.  ``My city has problems with its water supply.  I have heard that there are elves along the coast who may be able to help, so I intend to sail south from Westport to try to find them.''

``Fascinating,'' said Peter.  ``I don't see elves very often.  Occasionally in the city, but there aren't many that dwell in these lands---at least, not that I knew of.  I wish you good fortune in your search.''

