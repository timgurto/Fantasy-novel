\chapter{Silverdale}

When they reached the city, Tarn and Lawrence went their separate ways.  Tarn put his shield on his back and his hammer in his belt, holding his helmet under his arm.  He was happy to leave the remaining beer with Lawrence, who directed Tarn to the library before leaving for the town markets.

Silverhome was a medium-sized town of Men.  Fewer souls than Kobarthrond, he thought, but it seemed to cover a wider area---and that wasn't even counting the widespread farms outside the town proper.  Tarn marveled at the Human buildings: all free-standing, and constructed from wood, stone metal; whatever was most functional.  And they were tall: some two storeys high!  In Kobarthrond everything significant was carved into the rock, with the only free-standing structures being small things like tents or market stalls.  The technical achievement boasted by these Human buildings was impressive, but they also had a consistent failing: they were not beautiful.  Fit for purpose and well-built, certainly, but the builders clearly focused on utility and left symmetry, finishing and decoration by the wayside.  A cultural difference, Tarn supposed, which he must simply accept.

After the buildings, the next thing that caught Tarn's eye was all of the animals.  A shepherd walked along the road leading three large sheep. A man with a bow and a knife, whom Tarn guessed was a hunter, strolled along with a fierce-looking dog at his heel.  A knight wearing an elaborate plumed helmet and polished steel armour rode past on a well-kept horse.  At home, Tarn could go weeks without seeing an animal; aside from those slaughtered for meat, the only other animals he knew of in Kobarthrond were the chickens kept for their eggs.

In addition to the men with their animals, Tarn did spot the occasional Dwarf.  They were craftsmen, carrying special materials that could only be bought here, or trying to sell their wares.  They were exceptional though: most craftsmen stayed in Kobarthrond and waited for the merchants to come to them.

Following Lawrence's directions, Tarn found the library.  He knew enough of the Human tongue that he could understand the sign on the building, so he walked through the doorway.

Just inside the front door was a desk, staffed by an odd-looking person, the likes of which Tarn had never seen before.  He was tall, taller than most men, with skin pale and very smooth.  He had very long ears, and wore no beard.  Based on stories he had heard, Tarn could only assume that this creature was an Elf.

Waving, the person said ``Gr\=urg, tu ski,''.

``Grurg,'' Tarn echoed hesitantly, returning the greeting but surprised to hear his native tongue.

``Do you speak Human?  I know enough Dwarven to be polite, but I have never really had occasion to learn or practice it.''

``Yes I do, well enough I suppose.''

``Excellent!  My name is Bookie, and I am librarian here.  How can I help you?''

``Uhmm \ldots{} are you an Elf?'' Tarn stammered, trying not to sound rude.

``Yes I am.  A wood Elf, to be precise.  Am I the first you've seen of my kind?''

Tarn nodded.  The Elf seemed to have a strong sense of purpose whenever speaking or moving; deliberate and slow, yet elegant and efficient.  It was strangely pleasant to listen to him speak and to watch him work.

``There is no other in Silverdale, and so I may also be the only Elf that you ever see hence.''

After a moment of silence, Tarn came back to his senses.  ``Oh, err \ldots{} I've come looking for a solution to a problem befalling our city.  I was hoping the Men of this town might have a solution that we do not.''

``Your city being Orehome?''  Tarn nodded.

``I may not be a Man, but my position here is as a keeper of history, legend and truth.  In fact, I believe there is no other such record in the town.  Therefore, if there is a solution to be found in Silverdale, it may very well be in my library. If you would: what is the nature of Orehome's problem?''

Tarn told Bookie about the poisoned wells, the gut sickness, and that boiling the water is an apparent solution.  The librarian thought for a few moments, and then spoke.

``I have known of individual vessels of water becoming tainted, and the solutions I have seen are to boil the water---as you have said---or to discard it and fetch a fresh load.  I have not seen an entire well suffer from this that had previously been pure.

``That being said, I do recall an old legend.  It's not from this town, that may be relevant.  Please give me some time to try to find it.''

``Go right ahead,'' answered Tarn.

Bookie walked away from the desk, to shelves overflowing with books and scrolls.  They were stuffed into every available space, with no apparent system governing them, and yet Bookie seemed able to easily find anything that he was seeking.  It seemed strange that a creature exuding such discipline and control could be responsible for this mess.

After about fifteen minutes of searching, reading and cross-referencing, Bookie returned with a scroll.  ``This is the legend of which I spoke.''  He unrolled it and began to read:

**********
In the land of water, water was undrinkable because of a corruption from beneath the sea.
The people forged a sword, to smite that corrupting force.  Stab it.
Then the water is pure.

Amidst the memories of long ago
The land of water .....


Beyond the .. coast, above the waves,
by unknown arts the Land of Sea was grown,
while deep below, in ancient sunken caves
enchanted metals waited in the stone.

The Land of Sea was rock made smooth and true.
The architects a city founded there,
and from the coast a citizenry drew
who came in ships to see this vision rare.
who came in ships to prosecute the dare.
who came in ships to execute the dare.

industrious and active, ...
its citizens in ships came from the coast 


and though enveloped by the ocean .
but fit to drink?  The Land of Sea had none.



Sword
Water getting corrupted
Personified as demon, but plausibly literal
sword forged in the "land of water"
***********

Acknowledge that it sounds too fantastic to be real, but note that the real sword is a known artefact.