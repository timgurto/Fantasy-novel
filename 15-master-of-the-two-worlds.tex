Tarn knocked on the door, not knowing if Orvi would be home or out working.  He breathed a sigh of relief when the door opened.

``Tarn?  My goodness, it's been a while!  Come in, come in!  How are you?''  They sat at the table.

``Good.  Tired, but good.  I have just been on a grand adventure south.  But there will be time to discuss that later.  Tell me, how is the city's water situation?''

``I am not privy to the king's plans, but I can see with my own eyes.  I don't think the search for new water sources was successful; I would have heard about the plumbing work if they'd found anything.  We are still importing water, cartloads of barrels every day.  We haven't gone thirsty yet, but it doesn't seem sustainable.''

``Thank you for filling me in.  Now, please take a look at this.''  He withdrew \kildir\, and the chunk that had broken off from it, and put them on the table.

``It's a truly magnificent sword,'' Orvi said, staring into the blade, ``but I've never seen that alloy before.''

Tarn spoke quickly.  ``The metal is a mystery to me too.  This sword is called `\kildir'.  It is thousands of years old, dwarf-made, from the ancient city of \valdunmir.  When the metal touches water, that water becomes pure and unpolluted, no matter how polluted or stagnant it may be.  I have seen it work.''

``Really?  On what?''

``A frost elem \ldots'' Tarn realised that Orvi had never seen, or probably heard of, elementals before.  He quickly described it, and drew a picture in order to speed things up.

``And that elemental was filthy?''

``Revolting,'' Tarn confirmed, ``with mould and algae and dirt.  Disgusting.  And the sword cleaned the whole thing in a matter of minutes.''

``Then could this solve the city's problem?''

``I think it could.''

``Hold on \ldots\ \valdunmir?  What is that?''

``Legends called it the `Land of Sea'.  It was a floating city, ingenious and advanced, with fields and towers and thriving industry.  They too had a water problem, and they made this sword to resolve it.  This shield is also from the city.'' Tarn showed him the ceremonial shield.

``\ldots\ I don't know what to say.''

``Then I will do the talking,'' Tarn replied.  ``I want you to melt down this sword.  It works by coming into contact with the water, and so there must be a more appropriate shape.  I believe that if we insert the metal into the main well, we can purify the entire city's water supply.  The well is, what, four feet across?''

``Three feet,'' Orvi corrected him.

``Even better.  A shape should be chosen that will ensure the most contact possible between metal and water, and that will snugly fit into the well so that no water can escape it.  Will you take the job?''  He pulled out his gold coins and put them onto the table.

Orvi sat and thought for a few minutes, making some quick sketches, and examining \kildir.  He then picked up most of the gold coins and returned them to Tarn.

``I will do it,'' he said.  ``But I'll only take half payment, because this is a service to the city.''

``That is a noble gesture.  Oh, and if you can, please try to do justice to the craftsmanship of the sword.  It would be a tragedy to take something so beautiful and melt it down into something crude or ugly.''

``It goes without saying,'' Orvi said, smiling.  ``May I please hold onto this shield for now?  As inspiration''  Tarn agreed.

\divider

A few days later, there was a knock on Tarn's door.  Orvi was there, holding two wooden boxes.  He removed the royal shield from his back, and handed it to Tarn, who received it gratefully before saying, ``come in!''  They sat at the table.

Each box was built lovingly and expertly, and each was engraved with the same image of \valdunmir\ that adorned the shield.  \emph{Ever the craftsman}, thought Tarn happily.  Orvi handed him the first box  It was large and square shaped, with sides about three feet long, and a foot high.  As Tarn opened it, orvi said ``the filter.''  It contained a wide cylinder, filled with long tubes in a fine honeycomb pattern.  The cylinder was designed to fit inside the well, and the tubes would ensure that all water passes close to the metal, for long enough to be cleansed, before it can be drawn out and consumed.  Engraved on the smooth, round exterior of the cylinder was the following:

\dwarvenInscription{%
ointh orvi kogugim so\\*%
klointh tarn rolgugim so\\%
azinth akilb klum kildir grom\\%
sirk valdunmir bri\\*%
kildunt korbarthrondolb la tro
}

Translated, this means:

\settowidth{\versewidth}{Commissioned by Tarn, son of Rolg}
\begin{verse}[\versewidth]
Made by Orvi, son of Kog\\*
Commissioned by Tarn, son of Rolg\\
Forged from the legendary sword \kildir,\\
of ancient \valdunmir\\*
to clarify the water of \korbarthrond
\end{verse}

The bottom of the filter was flat, but the top was uneven, with hexagonal tubes sticking out unequally.  ``What is the pattern on the top?'' Tarn asked.  Orvi picked up the shield, and held it up behind the filter.  Then Tarn could see it:  the top of the filter was a model of the city.  Of \valdunmir.  He could see the island, and buildings and towers all around.  It was all a guess, of course, but the intent was clear.

``When the filter is submerged in the well, this model city will look like the real thing, under the sea,'' Orvi said, smiling.  All Tarn could do was smile back.

Orvi then gave Tarn the second box.  It was much smaller, a jewelry box that Tarn could hold comfortably in his hand.  He opened it.  Inside it was a small signet ring, made of the same teal metal from \kildir.  ``I kept it small, so that as much metal as possible could go into the filter.  It's in case you or somebody else wants to study the metal later.  And until then it can be a nice reminder of your journey.''  The inside of the ring's band was inscribed, like the filter, with Orvi's and Tarn's names.  The signet at the top of the ring was engraved in the likeness of the elemental that Tarn had drawn.  ``To remember what the metal is capable of,'' Orvi explained.

Tarn was delighted.  ``You are a superb craftsman, Orvi.  And an even better friend.''
