%\divider
\chapter{The Fight}

The first night had been uneventful.  They were now encamped for a second night, beside a large boulder.  They were close to the hills now, and so they avoided the open fields in the hopes of escaping detection.  Tarn was on lookout duty, and the black of night had just begun to fade into the dull dark grey of early dawn.

``Wake up!'' he whispered to Peter and Circe.  ``Somebody approaches; there are torches.''

They awoke, startled, and got to their feet.  Tarn pointed to the specks of light in the distance; flickering, bobbing around, gradually advancing.

``Should we arm ourselves?'' asked Peter.

``Yes, and be ready for a fight, but \emph{we will not be the aggressors}''.

The three of them prepared for battle.  Tarn donned his helmet, and held his hammer and shield.  Circe held her staff steady, crimson cloak rustling in the cold breeze.  Peter clutched the dagger in his right fist and a holy symbol in his left, and muttered quietly to himself in prayer.  Tarn was sorry for his friend, that they were going through this again after such a short respite.  It was good that this time they had an opportunity to prepare, but at the same time Tarn worried that the long suspense may worsen things for Peter, who struggled so much the last time they were sprung upon in the night.

At some point the group approaching must have realised that they had been spotted.  They now walked faster, more directly, and clearly made no further effort to conceal themselves.  Eventually, they came close enough that the two groups could see each other's faces.
%30m away

The newcomers were dwarves; four of them.  They wore the same leather armour that Tarn had seen on the corpse in the woods, whom Circe had killed.  Two carried bows, and each of the other two held a spear with both hands.

``Are you of the Sm\=athzolis?'' Tarn called out in Dwarven.

``We are,'' answered one of the dwarves.  ``We saw your camp while patrolling the border.  Why are you flanked by a man and an elf?  Who are you, and what are you doing intruding on our land?''

Tarn responded, ``My name is Tarn, son of Rolg.  I am from Korbarthrond, in the mountain range north of here, across the river.  My companions and I wish to speak with your leader on friendly terms.  We mean you no harm.''

``In that case, lay down your arms.  All of you.  Then we will talk.''

Tarn translated the conversation for Peter and Circe, who were understandably apprehensive about disarming in the face of armed, threatening---and as far as Circe was concerned, mortal enemy---strangers.  Peter in particular looked like he would never drop his dagger, such was the look of terror on his face.

Eventually, Circe whispered ``I can summon fire without my staff.  Let's humour them for now.''  All three lay down their weapons: hammer, shield, dagger and staff.

The scouting party approached, but Tarn spoke up and said ``not too close, with us unarmed.  We can talk from this far''

The advancing dwarves stopped, looked at each other, and seemed to conclude that it was a reasonable request.  ``Very well,'' one of them said.  ``Before we take you to Sm\=athzolunrond, you may speak with \emph{us}.  What is your business here?''

Tarn took a deep breath.  ``I have come to beg for a favour.  Korbarthrond is sick; our water supply poisoned by some corruption.  I have heard of an artefact in your possession, a sword that may have the power to save my city and my people.  I am here to humbly ask for that sword.

The scouts turned to each other and spoke quietly.  Some of them began laughing.  Suddenly, one of them raised his bow, and shot an arrow straight at Tarn.  Before he or anybody else could react, it pierced Tarn's left shoulder.  As he and his friends struggled to comprehend the situation, the two scouts with spears began to charge forward.  The other two scouts reached behind their backs for arrows, preparing to shoot more projectiles.

Snapping out of the initial shock, Circe reached down and picked up her staff where she had laid it down.  Peter crouched and felt around in the dark, but could not find his dagger.  Giving up the search, he instead ran towards Tarn so that he could help with his shoulder.  Tarn, meanwhile, had picked up his hammer, and was holding his shield up feebly with his punctured left arm.

``Stay behind me!'' commanded Tarn.  Peter obeyed, slowing his run and trying to stay low.  He had almost reached Tarn, when a second arrowhead emerged from the dwarf's back, having hit him in the stomach.  Tarn yelled in pain.  He glanced up to see what the other archer was doing, ready to block a third arrow with his shield.  Instead, Tarn saw him erupt with fire.  Circe had summoned a flash of heat and flame, centred entirely on that scout.  It lasted only a fraction of a second before dissipating into the cold, dark air, but it was enough to make the dwarf jump in pain and surprise, dropping his arrow and his bow into the grass and buying precious seconds without the threat of more arrows.

Peter scrambled to Tarn's side, and lay his hands on him while muttering a prayer.  A familar flash of light appeared, Tarn felt profound warmth radiating into his body from the cleric's firm hands, and suddenly he was healed: his shoulder and stomach were no longer in pain, the wounds had vanished, and the arrows that had been poking out of him had similarly disappeared.

The spearmen had almost reached them.  Tarn saw that Circe was wearing only her robe, and wouldn't stand a chance against the spears.  He needed them to attack \emph{him}, and not her.  He took a deep breath, and laughed ``I see Korbarthrond still makes the hardiest dwarves!  Your arrows can't kill me!''.  Whatever the impact of those words, the spearmen did indeed run towards Tarn.

Peter looked behind them, at the archers.  The burned one was coming to his senses and reaching for his bow.  The other had his bow raised, an arrow nocked, and the string stretched.  Peter pointed at him dramatically, squeezed the holy symbol in his left fist, and prayed.  Suddenly a sharp ray of sunlight appeared.  The sun had not yet become visible, but that didn't matter: the ray came not from the sun, but from Peter's outstretched finger.  It shot straight into the archer's face, and he abandoned his impending shot to cover his eyes from the bright light.  But there was nothing he could do to mitigate it: he was completely blinded.  Panic and frustration gave way to impotent rage, as the archer screamed and cursed, stumbling around and frothing at the mouth.

Tarn held his shield forward and his hammer raised as the two spearmen reached him together.  The one to his left thrust his spear forward, which Tarn blocked with his shield.  The other swung his spear down towards Tarn's head.  Tarn knocked it out of the way with his hammer.  While he recovered from the demanding parry, the first spearman also brought a spear down, this time making contact with Tarn's helmet.  It caused no injury, but the noise was deafening, and the tremor of impact shook his skull, making it hard to focus on anything.  The other spear, now effectively unblocked, pierced him in his right side.  Tarn exhaled violently.  As he felt the spear being pulled away, he grabbed onto it, dropping his hammer.  He was thus disarmed, but at least one of the Sm\=athzolis was too.  If he could defend himself from the other with his shield, he may survive the next moments.

The spearman on Tarn's left, who still had control over his weapon, glanced around to check for other threats.  His eyes rested on Circe, who had her staff raised and was summoning a large ball of fire in front of her.  While her initial blast was a quick reaction that didn't do much lasting damage, this seemed to be a much more demanding---and formidable---spell.  She was facing the archers, concentrating on her magic, and could not see the pending threat from the spearman who now watched her.  Tarn, again, felt like he had a better chance of surviving the spear than she did, and yelled to get his attention.  As the enemy turned his head back, Tarn swung his shield up into the dwarf's chin, then withdrew it before kicking him in the abdomen.  The scout clutched his stomach with both hands, dropped his spear, and fell to the ground, injured and winded.

The archer who had struggled to recover after being struck by that initial burst of fire now nocked a new arrow, ready to continue the fight.  Just as he raised his bow, though, Circe's fireball was finally released.  It flew across the field like a swooping bird of prey, leaving a trail of flame in the scorched grass beneath it as it passed, and hit the archer square in the chest.  He too fell to the ground, yelping in pain and rolling to extinguish the fire that had engulfed him.

Three of the belligerent dwarves were now incapacitated: one archer was blinded and the other burning.  The spearman to Tarn's left was on the ground nursing his injuries, and Tarn still grasped the spear of the other, holding it against his punctured side and resisting the attempts to withdraw it.  Peter, unarmed and scared, then stood.  He looked up into the sky, took a deep breath, and straightened his back.  He looked straight at the spearman, and spoke to him.

The words were loud; clear; profound.  Peter's voice echoed with a deep and otherworldly majesty that seemed to fill the sky.   ``You are an agent of evil.  You are falling short, wasting your potential.  But there is still hope for you.  Turn back to the light, fight for the good, and atone for your sins.'' It was impossible for the dwarf to resist the divine power carried by Peter's voice and words.  He turned to face Peter, staring, his eyes filled with tears of sorrow.  His mouth hung open in reverence and awe; he tried to answer but was unable to speak.  He let go of his spear and fell to his knees.

Tarn dropped the other end of the spear and knelt to pick up his hammer.  He raised it above the kneeling dwarf's head, and with a decisive blow he killed him.

Peter exchanged a small smile with Tarn, before saying ``the other dwarf can see now; I had to release him to focus on the Voice.''  Tarn looked up at the once-blinded archer, and saw that he was indeed coming back to his senses.  ``How bad is that?'' Peter asked, pointing at Tarn's injured side.

``It's alright.  Not urgent.  I can still fight for now.''

The recovering archer quickly found his bow, stood up, and saw Circe summoning yet another ball of fire.  He raised his bow, pulled back the bowstring, and shot an arrow straight at Circe.  She fell breathlessly as the arrow went into her chest, and the growing orb of fire harmlessly evaporated as her focus was lost.

Peter cried out when he saw the arrow strike Circe.  Without hesitating, he knelt down, raised his arms, and closed his eyes, chanting in a low and quiet voice.  Tarn could feel the healing warmth again, but this time it didn't come from Peter's hands; it seemed to be growing from within him, in his chest.  It stretched out, until suddenly his whole body was bathed in light, just as if he had been standing beneath a sunny sky.  He looked over to see that both Peter and Circe were covered in similar auras of light.  Peter stood up, squinting from the sudden brightness, and called out ``they cannot penetrate this holy protection!''

While Tarn was distracted by the spectacle of this new magic, the other spearman, who had been lying on the ground nursing his injuries, stood up with his spear.  Tarn saw him out of the corner of his eye, but before he could face his opponent, or raise his shield or hammer, the Sm\=athzolis thrust his spear at Tarn with all his might.

\ldots\ and nothing happened.  The spear bounced off of Tarn's skin violently, as if it had struck a large, solid rock.  Tarn felt no pain or force from the blow.  He was astounded: if this was Peter's holy shield, then it was a powerful thing indeed.  Hesitating no longer, he stretched back his hammer and struck the confused spearman in the chest, knocking him down.  He then struck him twice more.  The two archers alone now remained.

Circe, arrow still in her chest, struggled to stand and to breathe, but she managed to do both.  While defended by the magical shield, she took the opportunity to cast yet another spell.  Ignoring the pain in her lungs and the shallowness of her breathing, she held her staff out horizontally in front of her with both hands.  She took a deep breath, closed her eyes, and focused entirely on her magic.  About fifty feet in front of them, a line of glowing embers appeared in the grass.  It grew longer, and thicker, and suddenly tall flames jumped up out of them and reached the sky.  An enormous wall of fire now lay between the two archers and everybody else.  It was too long to go around and too high to see through.  The archers could do nothing.  Circe continued channeling the spell; it seemed that maintaining the fire wall required ongoing effort and focus.  Meanwhile, her breathing became yet more laboured, and she grimaced in forced concentration and in pain.

Peter ran to the pyromancer urgently, though he was clearly exhausted  Tarn walked to them, clutching his side.  The shields surrounding them faded away, their protection ended.  Peter held his hand over the arrow wound in Circe's chest, and with a prayer he healed her.  He then turned to Tarn and did the same with the spear injury in his side.  Peter said ``I need a rest.''  He pulled out his water skin, and had a long drink while Tarn walked over to him.  He then sat, eyes closed and holy symbol still clutched in his hand, and he was silent for about a minute.  When Peter opened his eyes, he stood to speak with Circe, who continued to maintain the fire wall.  He said, gently, ``when you need to stop, do so.  Have a rest.  Meditate, and recover your strength, your focus and your courage.  We can handle ourselves against the remaining two while you recover.''

Tarn raised his shield, and Peter stood behind him.  They stayed close to Circe.  Soon she opened her eyes, dropped he staff, and collapsed to the ground.  The wall of fire died down quickly.  On the other side, the two Sm\=athzolis archers were ready.

The archer on the left had burn marks on his face and left arm, and smoke rose from his hair and clothes.  The one on the right was uninjured, but deeply frustrated by having been blinded by holy light and then caged by magical fire.  Both archers were furious.

The unburned archer on the right yelled, ``you have no idea what you are meddling with, dwarf.  We are righteous.  We are restoring a great legacy.  Your dirty mountain tunnel will be a footnote in the history books.  Our victory will fill a library!''

Tarn had no time to be offended.  His priority right now was to keep the archers watching him, and not Circe or Peter.  He replied ``I didn't even know you were here, until some elves told me!  How notable could you possibly be?''

The archer who had started this conversation seemed suitably annoyed with Tarn now.  The other, burned one still had his eyes fixed on Circe, the cause of his pain and torment.  He shot another arrow at her.  Tarn, ready, lunged over towards the elf and held out his shield.  As he fell to the ground, the arrow struck it and bounced off.  Tarn stood quickly, shook Circe by the shoulder and yelled that she was now vulnerable.  She opened her eyes.  Both Circe and Tarn suddenly heard a man's cry, and turned to see that the other archer had loosed an arrow straight into Peter's stomach.  He now lay on the ground, doubled over and groaning.

``What can we do?'' Tarn quickly asked.

``Nothing.  Don't worry, I'll be okay,'' came the laboured reply.  ``Go!''

Tarn and Circe charged.  The archers were about 100 feet away; too far to reach before they could ready their bows again.  Circe suddenly stopped, held up her staff, and conjured a small projectile of red flame.  It shot out of her staff, struck the ground between the two archers, and knocked them both over.  The spell seemed to contain little heat, but plenty of force.

Tarn, who had continued running, quickly reached them.  He hit the burned archer with his hammer, killing him.  He then turned to the the last surviving scout, pinned his neck to the ground with his shield, and said, ``let's talk,''.  As soon as the archer opened his mouth, he was consumed by intense fire, summoned by Circe.

``We could have gotten information from him,'' Tarn said, annoyed.  ``He was disarmed.''

``He was a zealot, and wouldn't have given up anything useful.''  the elf replied.  ``Besides, he could have had a hidden knife.  Hey may have thrown a rock.  There might be others around to assist him.  You are too careless.''

Tarn was an experienced soldier, a guard of the city.  But his training was for pitched battles and manning ramparts; he was not accustomed to small skirmishes and dirty fights.  So he felt he had to let this point go.  They already knew where they were going, in any case.

He suddenly remembered Peter.  He turned around to run, but saw the cleric walking towards them.  He looked like he had no emotions left: dark and deep eyes, no smile or frown or grimace in his mouth or eyebrows.  The fight had obviously taken a toll on him.  But he was alive \ldots\ and physically healthy.

``You could heal yourself?'' Tarn asked, surprised.

Peter quietly replied, ``it's the first thing we learn.  And the only real way to practice.''

Tarn picked up a bow and a quiver of arrows, and handed them to Peter.  ``This should replace the hunting bow you lost to the goblins,'' he said.  He also took a spear for himself.

``I lost the dagger,'' Peter said, ``and I don't care much to go looking for it.  It did me no good today, and all it does is remind me of death.''  Tarn looked at him sympathetically.  The dwarf was trained for combat, and death, and killing.  Peter was not.




The three walked back to their camp.  Violence was the only option left to them now.  As the sun peeked out from the eastern horizon, they started a fire, and sat down to eat, rest and regroup.

``That was some very impressive magic,'' Tarn said to Circe.  ``I had no idea that fire could be controlled so precisely, or that fire magic was so versatile.''

``I have studied pyromancy since I was a little girl.  The Sm\=athzolis burned down my village; all my childhood memories are saturated with the glow of flame and the stench of smoke.  I decided that I would become a mistress of fire, that I would learn how to bend it to my will, so that one day I could have my revenge on the dwarves who took my home.  I wanted to do to them what they did to me.''

``And now you have done so,'' said Peter, ``at least with a scouting party of them.  Was it as satisfying as you dreamed it would be?''

``\ldots\ it was a good start,'' she replied.



