\chapter{\mothzam\ \driktur}

Tarn, Peter and Circe left the elemental, closing the door of the room behind them.  There were a number of forks in the tunnel that they had passed when following the trickling water, and one of them probably led to \mothzam\ \driktur.  Circe's staff was still giving off its grey light, and she led the group as they began exploring.  The tunnel floors along all paths seemed covered by those same stagnant puddles and patches of green growth.

As they turned one corner, a guard suddenly ran out at them with sword in hand, yelling.  Tarn instinctively leapt in front of Circe, grabbing his shield but having no time to draw his hammer.    The guard swung his sword, and Tarn managed to block it with his shield, taking the opportunity to grab at his belt for a weapon.  His hand found \kildir, and without thinking, he withdrew it and swung.

The guard quickly parried.  The two swords met, and the sound of metal on metal rang out.  Tarn could feel his sword being moved, dragged away, and he glanced at it.  Where the swords had touched, a gap now lay in \kildir's blade, a large chip having flown off somewhere.  Tarn screamed in anger, and in pain for his damaged treasure.  Suddenly fire consumed the guard's chest and pushed Tarn backwards, onto the ground.  He looked up to see Circe's staff outstretched towards the guard, now dead and smouldering.

Tarn collapsed onto his knees on the damp stone, and wept.  He had suspected that the sword was not intended for combat against other arms, and now that had been proven in a devastating way.   Not only was the sword in his hand severely weakened---making it even more defective as a weapon than it already was---but its appearance was now disfigured, with a large chunk missing.  Upon viewing it, any person would no longer see a beautifully crafted artefact deep in history and myth, but instead merely a damaged hunk of queer green metal.  His imagination been possessed with visions of owning this treasure, and it was as if his entire personality had become defined by that ownership.  And now those visions were debased and distorted.  He felt empty.

\emph{Maybe it can be fixed}, grasped his desperate mind.  If he could find a good blacksmith, then maybe the sword could be repaired.  He didn't know anything about the metal.  And in its broken state, \kildir may have lost whatever power it had held---the power to purify an elemental or to save a city.  But he had to try.  Willing to entertain any solution his imagination could latch onto, he began searching the tunnel floor for the chip that had broken off.

Eventually something caught his eye.  Among all of the puddles of fetid brown water, one and one alone seemed to contain clear, clean water.  He threw himself down to the ground there, and found the chunk of teal metal lying in the puddle.  It had plainly cleansed the puddle, just as the sword had cleansed the elemental.

\emph{It's the metal}, Tarn realised.  If a chunk could fly off and still retain the same power, then the magic wasn't in the sword, it was in the metal.  And there was no ritual necessary; it just worked: the metal purified whatever water it touched.  The song he'd heard in the library had convinced him that the sword had slain a demon---obviously its author had taken some poetic liberty when describing the effects of this metal, saying that the hero `smote the demon, watching it disband', and other such language.  Tarn gave a small smile.  The truth was more mundane.  Yet it was also more practical.  He picked the chunk of metal up out of the puddle and held it, thinking.  \korbarthrond's problem \emph{could} be solved, after all.

``Are you alright?'' Peter asked.  ``How badly damaged is the sword?''

``The magic is in the metal,'' Tarn answered, ``and the metal is all still here.''