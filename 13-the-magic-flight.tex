Then the elemntal asked, ``where does your quest take you now?''

``We return now to Treeville, to brief lord Elrond about what occurred here, and to return to her people the belongings and story of our fallen friend.''

``There is a river delta near these hills, which leads upstream to Treeville.  I can travel swiftly and powerfully in such waters, and I would be honored to carry both of you there.''

Tarn and Peter exchanged a look, then Tarn shrugged and said, ``that sounds like an unmissable opportunity.''

There were three companions now, advancing up the gradated tunnel, up the stairs then out the gate of the hill fortress.  The air was fresh, but chilling; it was the coldest part of the night, just before dawn.  Tarn and Peter crept out and into cover in case anybody in the village was watching, while the elemental followed them as quietly as he could.  The constant sound of agitated water followed him, as parts of him splashed to the ground before coalescing back into him.  They quickly headed west, out of the hills, until they reached the river delta.

``Wade into the water,'' instructed the elemental.

Tarn and Peter did so gingerly, bristling at the cold water and careful to hold their belongings tightly.  The elemental followed them into the river, crying with delight at the sensation.

``It has been so long!  So long since I have felt other water.''

He soon returned to his normal humourlessness, and continued his advice.  ``Hold your breaths.  When you are accustomed to the movement, and the water in your faces, you may trust yourselves to breathe again.''  Tarn and Peter, now shivering in the cold water, took deep breaths.  The elemental then moved behind them, and suddenly a great force pushed them forward, like a wave that never broke but instead kept advancing.  After a few moments they began breathing again, careful to inhale air instead of the water that rushed constantly at them from all directions.  With the elemental pushing them from behind, and the water in front of them pushing against the movement, they struggled to hold onto everything.  Looking at the nearby land, Tarn saw that they were moving at phenomenal speed, at least twice as fast as a swift horse, and faster than any running water he had ever seen.  In less than an hour, they saw Treeville appear in the distance, and very soon after that, as the pink light of dawn began filling the sky, they found themselves in the depths of the elven woods.

\chapter{Reclamation}
``Go on into the village without me, my friends,'' said the elemental.  ``Though it would please me to meet with this elf lord, I cannot yet bear to leave the water.  I will linger here until dawn tomorrow, in these slow and clear waters, so that you may speak with me again if you so wish.''

Tarn and Peter bowed to the elemental, then turned to enter the village.   They were soaking wet, and shivered as the morning wind hit them, yet they were still so thrilled by the rapid journey they had just taken that they hardly noticed the cold.  The leaf cover crunched under their feet as they reached the large tree in which they had twice met with Elrond.

The guard at the base of the tree saw them coming.  ``Are you the man and the dwarf who rested here recently?'' he asked in broken Human.

``We are,'' replied Peter, faring better in his native tongue than Tarn would, just as when they dealt with the elves before.  ``We seek an audience with Lord Elrond.''

The guard nodded, then moved aside so that they could enter the tree and begin climbing.  It was a slower ascent than last time, as Peter wrestled with how he would explain what had happened to Circe.  Finally they reached the high branch with the meeting hall, and therein was Elrond, sitting again at his table.

``Tarn, son of Rolg, and Peter of Silverdale.  It is good to see you returned,'' he said, ``but I see that there are only two of you.  What became of Circe, the dark elf who accompanied you southward?''

Tarn answered, ``She came with us under the suspicion that the Tholkis were undertaking something dark.  She was proven correct.  We confronted and defeated their leader, \mothzam\ \driktur, but Circe died in the conflict.''

``That is troubling to hear.  Every immortal soul is a special treasure in this world, and to lose one is a significant loss.  That loss is compounded by Circe's being one of the few remaining dark elves from Swampville.  Her kin will be especially sorrowful upon hearing this news.''

``She fought bravely for the good and for the sake of those in need.  Her life, and her remarkable skills in pyromancy, were not lost in vain.'' Peter said.  Then, remembering Circe's possessions, ``these are the only items she had, other than her staff which was burned with her body.''  He offered Elrond the codex and the purse.

``I will return these to her people,'' Elrond said, ``but there is more of your story to tell.''

At this, Tarn described their entire adventure: the fights they had with Tholkis guards, Tarn's vision of the divine voice, finding and using \kildir, their fight with \mothzam\ \driktur, and the elemental who was freed and now waited in the river below.  Elrond asked no questions until they were finished, interjecting only to clarify what was said.

``I do not comprehend the mysteries of your religion of the Light,'' Elrond said to Peter, ``but it appears that your friend was blessed in his quest.''

``I truly believe so,'' Peter replied.  ``It was equally humbling for me as Tarn's companion.''

Elrond turned to Tarn.  ``As for that quest, I am pleased that you have recovered the stolen sword, and moreso by your confidence that it will help Orehome with its drinking water.''

``Not nearly as pleased as I am,'' Tarn replied.

``It was also good that you purified the frost elemental, even if that was not your original intention.''  Tarn grimaced, embarassed.  ``Though the intent and the outcome differed, both were noble.  To be free from years of imprisonment and torture, it is no wonder he granted you the gift of swift passage here.  Have you learned much about him, his kin, or what he intends to do now?''

``I have not.  The elemental is not very talkative, and we had little opportunity for idle chatter.''

``A pity.  Perhaps I will have a chance to converse with him while he remains in our waters.

``As for \mothzam's plan, I agree with Circe's assessment that resurrecting the city, if possible, would have been a move away from balance.  Nature conspired to destroy Sinkopolis, and to defy that clear act of volition---for there is a will to be found in the natural world---was always a doomed prospect.''

``Now that their leader is dead, what will become of the villagers in Tholkunrond?'' asked Tarn.

``I cannot say.  From your description it seems like a functional village, and it may thrive even without the grand plan of \mothzam\ \driktur---perhaps especially so.  All we can do is observe them going forward.  We will of course defend our woods if they should attack, but until that eventuality we will trust them to be peaceful neighbours.

``Now, may I please inspect the artefacts you recovered from \mothzam's chamber?''

Tarn produced the shield that had hung on the wall, and handed it to the elvish lord.

``This shield is from antiquity,'' Elrond said, turning it over in his hands.  Based on your translation of the inscription, I would say it did indeed belong to Prince Skamzold of Sinkopolis.  It appears to be ceremonial in purpose, and not intended for combat.  The metal is a special alloy of iron that will not rust, though that comes at the cost of strength.''  He handed it back to Tarn.

``Thank you, my lord.  It seems that special alloys were a favourite of that city.''

``Indeed.  And the staff?''  he asked Peter.  The cleric gave it to him.

``Dwarf made,'' Elrond quickly concluded.  ``\mothzam\ \driktur\ was a cryomancer, a wizard skilled in summoning and controlling ice.  His possession of the elemental's core essence enabled deep study, and he drew on its power---power that a wizard normally cannot have.  Given how powerful a wizard he was, it is unsurprising that your battle with him was a deadly one, and it is commendable that you survived it at all.

``I believe \mothzam\ created this staff himself.  It was designed to contain that essence, and to channel its power.  The essence is now gone from the staff, but some of its residual power is retained.  This is power that you can learn to use, cleric Peter.''  He returned it to the grateful man.

Elrond continued, ``I also see \kildir\ on your belt, Tarn, but our records still accurately describe it, and so there is nothing more that I can tell you about it.  You have the permission of my people, and of the dark elves from whom it was stolen, to return with the sword to Orehome.''

``Thank you kindly,'' Tarn said.  ``I will not forget your generosity in this matter.''

``And I thank you, for allowing us to make our records and legends more complete.  We will document that the sword has passed to Orehome in the north.  Do you intend to travel back there soon?''

``As soon as possible.  Although first I wish to visit the Silverdale library once more, so that Bookie too can fill in the gaps in his records.''

``A kind thought.  If you would allow me some time, I will prepare a gift for the esteemed librarian so that you can take it to him.''

``Of course,'' Tarn answered.

``And you, Peter?  Will you seek to travel on to Westport, as you had originally intended?''

``If possible, my lord, I would like to stay here in Treeville for a short while.  It is a rare opportunity to discover other ways that the Light manifests.  I would also like to speak directly with Circe's peers about her courageous actions and glorious death.''

``You would be most welcome,'' Elrond said, smiling.  ``Thank you both for sharing your story with me.  Now I must leave you to the hospitality of our village.''  And with that, they left the meeting room and the tree.

They spent the rest of the morning sleeping, in that same guest house in the small tree.  In mid afternoon, a young elf arrived and awoke Tarn.

``I am a messenger from Lord Elrond,'' he said, ``and I have items for you to give to the librarian in Silverdale.''  The elf handed him a scroll and a small pouch.

Tarn thanked the messenger, and he left.  The pouch appeared to be full of seeds, while the scroll was indecipherable: a long text written in the curved elven script, attractive in its way but completely alien to Tarn.  He decided to visit the elemental once more while he could, before making preparations for his return to the north, and he climed down the tree and headed to the river.

``Hello again, my friend,'' said the frost elemental as he saw Tarn approaching.  A small crowd of elves surrounded him, enjoying the novel sight.  ``I have spoken to the elf lord.  We discussed much, and learned many things about each other's people.  I thank you for giving me that opportunity.''

``Not a problem,'' Tarn replied.  ``I will be returning home soon, and so I wanted to say goodbye.  It has been a long and tiring journey, but meeting you was something that I will never forget.''

``That is kind of you to say.  To be remembered is a fine thing.  To which `home' will you return?''

I come from Korbarthrond, Orehome, in the mountains to the north.  But my immediate destination is the town of Silverdale, near to that.

``If you so desire, I would gladly convey you there, just as I brought you here this morning.  The service would pale in comparison to those that you have rendered me.''

Tarn was surprised, but replied, ``I would be foolish to refuse!  Please let me go and gather my things.  I will be back in half an hour.''

Tarn hurried back to the guest house, and collected his clothes, the scroll and pouch for Bookie, his shields and hammer and helmet, and \kildir.  The scroll and pouch he wrapped in a cloak to keep them as dry as possible for the journey.  He then woke Peter to say goodbye.

``It has been quite a journey, Tarn,'' Peter said.  ``I hope you keep your necklace well.''

``I will think of you whenever I see it.''

``Make sure you think of the Light too,'' he chuckled.

``I wish you a good time with the elves.  May you learn much.''

``Thank you. It is a truly magical place, and I find myself more content here than I have been in a long time.''

``What, more content than when sleeping in the mud?'' Tarn asked.  ``Or fighting for our lives outside the hills of \tholkunrond?''

Peter smiled.  ``As difficult as those challenges may have been, I will be forever grateful that they provided me with a good friend.''

``And I too.  I hope that you visit Orehome one day.''

``I will.  And on that day, I hope that we can share a drink of water.''

``I pray that we can.''

They embraced.  Peter then said, ``do not tarry.  Your city needs you.''

Tarn picked up his things, smiled at Peter, and left with a heavy heart.  He hurried back to the river, and the elemental greeted him.  ``Are you ready to leave?''

``I am.''  Tarn waded into the water as before, and held his possessions tightly.

``This will be a longer journey than before.''  Tarn nodded, and held his breath.  With a sudden sound of rushing water and a tremendous pushing force, they were off.

Tarn didn't know whether it was possible to reach Silverdale by this river, but the elemental seemed confident in his directions.  The river wove left and right, meandering and joining forks, widening and narrowing, and it was impossible for Tarn to keep his sense of direction. Focusing on the water was sure to make him feel ill, so he instead watched the distant landscape go past.  He saw a number of settlements on the river: towns of Men and camps of Goblins, of many different sizes.  He imagined \valdunmir, and how glorious a city that must have been, with its high-reaching towers, and sparkling seas, and wonderful metals.  None of these river towns came close to the floating city of his imagination.  And in every region in the land, and even on the other continents, he found it hard to believe that any place could ever be so impressive.  As the river water sprayed his face, he couldn't help but wonder if ther was ever really a chance of restoring \valdunmir\ from the depths of the sea.  \mothzam\ was surely mad, but Tarn quietly sympathised  with that madness.





















