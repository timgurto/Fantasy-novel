The three companions then left the study and returned to the main hall.  It was straight and level, with a small window in the distance allowing a dash of the afternoon sun to overpower the torches lining the walls.  Moving down the hall, Tarn spotted a doorway to the right, and poking his head through he found staircase going down.

``Shall we bring a torch?'' he asked, gesturing at the nearest one on the wall.

``No need,'' answered Circe, holding up her staff and muttering until a small flame appeared on its tip, light grey in colour, looking almost ghostly.  ``It isn't very hot, but it's bright.'' Circe then led the party carefully down the stairs.

It led to another hallway, this one completely underground.  Its floor, walls and ceiling were raw stone---really more of a tunnel than a hall.  \emph{Rough and undecorated}, Tarn noted with distate.  It stretched in only one direction, and inclined downwards.

They began following the tunnel.  Peter and Circe were careful not to hit their heads, but it was mostly tall enough for them to walk comfortably through.  The air down here was thick and still, filled with the smell of mould.   Staining the ground were damp patches and growths of green algae, growing larger and more frequent as they walked.  Soon there was a thin stream of brown water trickled through the middle of the tunnel, glistening as Circe's grey firelight danced on her staff.

``Wait,'' Peter said.  ``Look!''  The others stopped and turned to see him looking down, pointing at the ground.  They followed his gaze.  ``It's flowing uphill, back the way we came!''

Tarn and Peter looked at each other.  ``Do you think \ldots'' Tarn began.

Circe looked puzzled.  ``It's similar to the water we saw in the river, before you found us,'' Tarn explained.  ``Flowing in strange directions instead of following the grade of the land.  Lord Elrond said it may have been a frost elemental.''

``Let's find out where it's coming from,'' said Peter.

Circe nodded, and they followed the trail of liquid deeper through the tunnel.  It took them around bends and through forks, as the stream gradually thickened, and a rumbling sound began to fill the air.  Eventually it brought them to a large wooden door.  The doorway was carved crudely into the stone, and the door fit very roughly, such that there were gaps all around it, and plenty of room at the bottom for the water to creep through.  The stagnant water was definitely flowing out of this room, and an eerie green light could be seen through the gaps.  The room also seemed to be the source of the noise.

Tarn reached for his hammer and shield.  ``Are you ready?'' he asked.  His companions assented, and he gave the door a hard push with his foot, opening it.

They beheld a mass of liquid, about twice the size of a dwarf.  It was held together in the vague shape of a person, with a head and arms protruding from a wide base.  It appeared to be that same stagnant water: brown and cloudy, with green algae marbled throughout it.  The water was moving in many directions at once, some falling and dripping naturally from the arms and extremities to the ground, and some gathering into the base, coalescing and flowing up into the rest of its body.  It made the thing pulsate, its shape broadly holding together while any given part of it was in constant flux, growing, shrinking and moving.  In addition, the whole being appeared to thrash around, splashing even more water onto the ground.  The fetid smell of mould and decay was very strong, filling the small chamber.

The thing was screaming.  Or so Tarn assumed; the sound was like the loud rushing of a waterfall mixed with the gurgle of magma, and he thought he saw the creature's head contort with a mouth-like cavity appearing briefly, before being subsumed by the flowing water.

Despite its otherworldy appearance, the mass of liquid did not seem to pose any immediate threat.  It stayed in its place and made no moves to approach or reach towards the onlookers, instead focused on its own movements.  Tarn felt it was safe to turned away from it, and he conferred with his friends, yelling over the terrible sound.  ``Have you ever seen anything like this?''

``Never,'' returned Circe, mouth agape.  Peter simply shook his head.

Tarn's thoughts wandered back to the Silverdale library, and \emph{The Land of Sea}.  ``The legend of this sword spoke of it slaying a demon that corrupted the water around Sinkopolis.  Could this be such a demon?''

He glanced around at the room.  It seemed to be a natural cave, raw and unshaped.  There was no way in or out other than the doorway in which he stood, and nothing else in the room.  It was unclear how the demon got here or why it stayed; there was no water for it to corrupt.

Tarn withdrew \kildir\ and considered his treasured sword.  Maybe it could slay the demon; maybe not.  Maybe it needed to be used in a certain way in order to do so.  Maybe if he attempted it, the sword would break, or be absorbed within the demon, or itself be corrupted. ``I don't want to risk the sword.''

``Did you hear it?''  Peter asked.  ``That noise it makes is horrible.  Evil emanates from it.''

``Even so, this isn't my fight,'' Tarn said defensively.  ``In any case, I still don't know how to use this.''

Circe said, ``we should keep exploring these caves, and find \mothzam\ \driktur.  He was your best plan for learning more about the sword, and he still may be able to give us that information.''

``But what if this thing is \mothzam's servant?'' asked Peter.  ``All we know about this place is that \mothzam\ is here, and so it's likely that this demon has something to do with him.  If we can slay it then let us do so now, while it is idle.''

Tarn had considered leaving after he found \kildir, but ultimately his sense of duty to a larger cause---fulfilling his potential, as the cleric would put it--- led him to choose delving deeper into this hill.  Surely that same reasoning applied here, too?  He was a mere guard on a quest to help his city.  And even then, he found himself driven more by greed than by that civic responsibility. He was no hero. But despite that, Tarn now found himself in the very same room as a thing of evil, and held in his hand the means to defeat it.  Above all, he was convinced by Peter's advice to dispense with it now, if possible, while it was not aggressive.

He asked the cleric, ``Can you give me another of those holy shields, like you did in our fight against the guards?  If the sword doesn't work, and the demon moves to attack, I'd like to be protected.''

``Of course.''  Peter knelt and raised his arms, chanting as before.  Tarn felt the light growing inside him, until it covered his whole body.  Now that he could focus on it instead of on the dangers and bloodlust of battle, he realised just how warm it felt; very much like the feeling he had when the divine voice had spoken to him.

Thus protected, Tarn stepped towards the demon, sword in hand.  He approached cautiously, in case it realised what was about to happen.  When within reach, Tarn gripped the hilt tightly, and in a quick, firm movement he plunged \kildir\ into the creature's chest.

There was a scream, just as before.  The sword rapidly became freezing cold, until Tarn could no longer bear to hold it.  His fingers loosened and he took a begrudging step back, while the sword stayed in place.  Tarn stared at the place where he had stabbed the thing.  At the point of the wound---if such a thing could indeed be wounded---the colour of the water was changing.  It became a clearer brown, as dirt and muck seemed to be forced outwards into the rest of its body.  Then it became clearer still, and paler, until the water all around the blade looked like clean, fresh water.

Tarn continued staring---\kildir's magic was working!  It took this filthy, polluted thing, and was turning it into pure water!  Tendrils of clear liquid started spreading out from the blade, slowly, like a black poison might spread under the skin from a venomous bite.  As the clear water in the middle spread outwards, dirt and mould and other filth was driven out to the surface and the extremities, where it was pushed out before sloughing off onto the ground.

By now the creature was about half clean.  Suddenly the water immediately touching the blade changed; it rippled in a strange way, and then crystalised into ice.  As the purity spread through the filth, the ice followed, spreading through the clean water.  Soon the screaming died down, until the only sound in the room was the gushing of water and the occasional plop of muck hitting the stone.  The thrashing also stopped, replaced with the rhythmic pulsing of water throughout its body.  As the ice continued to spread, the movement slowed and the sound grew quieter, until eventually the whole creature was clean, and the fluids making it up seemed like a half-frozen stream, moving gradually yet purposefully.  It was mostly quiet now.  And then a voice echoed through the room.

``You have cleansed me; I thank you.''  That same mouth-like hole that seemed to be screaming before now spoke, in a voice low and gurgling, but comprehensible.  The words were in the Human tongue, which Tarn considered odd for such a non-Human creature.  Then again, that language did seem ubiquitous in this region.  Even he, who before this adventure had lived a quite insular life in and around \korbarthrond, spoke Human.

The creature's arms moved forward, and seemed to grasp the hilt of \kildir.  It pulled the sword out of its chest, and dropped it onto the stone floor with a violent clang.  \kildir\ seemed to have survived the transformation: it was whole, and clean, and looked uncorrupted.  Tarn reached forward to pick it up, but found that it was still too cold to touch.

The voice continued, ``I do not intend to hurt you.  I am a warden of the frosts and waters of the world.''

``\ldots\ an elemental?'' asked Tarn, shocked to be speaking so cordially with the very thing that he had moments ago intended to kill.

``Yes, that is the name given to our people.  I am a frost elemental.''

Tarn was relieved to be dealing with such a monstrous entity so amicably, but there was still the question of its loyalties. ``What are you doing in this place?''  

``I am a prisoner, trapped here for many years.  With no fresh water to draw on, I stagnated, gradually consumed by dirt, and mould, and algae, and other growing things.  I could hardly move or speak, and now, after all this time, you have cleansed me.''

By this time Peter was satisfied that Tarn was safe, and so he allowed the holy shield to dissipate.

``Who imprisoned you?  Why?''  asked Tarn, intrigued.

``I was captured by the dwarf \mothzam\ \driktur.  He desires to exploit my influence over water, to raise a city from the bottom of the ocean.''

``\ldots\ Sinkopolis?''

``Yes, that is the name of the city.  Although \mothzam\ calls it \valdunmir.''

``The sunken city,'' Tarn muttered, translating the Dwarven name.  He was flabbergasted. ``What is his reason for wanting to do that?''

``I do not know,'' answered the elemental.  ``But I cannot help him.  I will not.  The city fell beneath the waves thousands of years ago, and there it now belongs.  It is not the place of me, nor \mothzam, nor any person to undo what was done by such powerful and purposeful forces, nor to overturn a state of affairs that has been in place for so long.  We are masters of the storms, not of other peoples or their cities.''

``Now that you are restored, will you leave?'' asked Peter.

``I am myself once again, pure and unimpeded by the filth and growth of years, but still I cannot leave.  My essence has been taken from me.  It is a small thing, but without it I cannot seek new waters.  \mothzam\ holds it in a bottle, and refuses to return it to me until I perform the task he desires.''

``We will retrieve it for you,'' said Circe.

``Hold on,'' Tarn interjected.  That seemed uncharacteristically impulsive for the elf, he thought.  He asked the elemental, ``Do you know where he keeps it?''

``To my knowledge, it is always on his person,'' came the gurgling reply.

``We need to confront \mothzam, in any case,'' Circe asserted quickly.  ``I share the view that his plan should be stopped.  A dead city cannot be made alive again, and there is nothing to be gained from those waterlogged ruins.  It is perverse to act in such discord with the forces of nature that destroyed Sinkopolis in the first place.

``In addition, regardless of his purpose, \mothzam's imprisoning an elemental for \emph{years} in order to extort it is a significant crime, and I cannot abide it.''

Tarn looked at the cleric.  ``Peter?''

The man shrugged in response.  ``I have no interest in Sinkopolis, one way or the other.  But I agree with Circe, that he should answer for the suffering caused to this elemental.''

Tarn let out a sigh.  ``Very well.  We will try to find \mothzam\ \driktur, and he will answer for this.  He tried again to touch \kildir, and finding that it was no longer quite so cold, he picked it up and tucked it back into his belt.


















