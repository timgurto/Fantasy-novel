\chapter{Treasure}

The fortress was cool inside, with a breeze gently passing them from deeper within the hill.  The main hall went straight forward, into the hill.  It smelled damp, and old.  Tarn guessed that this was only the surface level, and that below it there was much more to be seen.

Once inside the entrance he looked to the left, as advised by that divine voice, and saw a short passage leading to a wooden door.  He led the party down the hall, with Peter and Circe ducking under beams and taking care not to hit their heads on the dwarf-built ceiling.  Opening the door, Tarn found a small study, just as the voice had described.  It contained many books, tablets and scrolls, as well as a number of maps hanging on the walls with varying amounts of annotation and sketches upon them.  There was a desk too, cluttered with documents and opened books, and samples of rock and metal that appeared to be under study.

On a shelf on the far wall, Tarn saw a glint of light and the shiny surface of a polished, teal-coloured metal.  His heart began to beat faster as he walked towards the shelf.  The object was under a pile of papers, which Tarn carefully removed and placed on the floor by his feet.  He reached up and grabbed the metal object.  It was a sword.

``Is that it?'' Circe asked quietly, reverently.  ``\kildir?''

Tarn beheld it.  It was indeed teal colored; polished but not very sharp.  It was fairly long for a dwarf, reaching from about his waist to the floor.  He could hold it up without much trouble; whatever this alloy was, it didn't have much heft to it, and Tarn doubted it would perform well in combat except against an unarmed and unarmoured opponent.  The hilt was not wrapped in leather; perhaps it had deteriorated long ago, if the sword were as ancient as he hoped it was.  There was a pattern finely engraved on the blade, a motif of water drops and flowing lines---different than the straight edges and geometric patterns common to dwarf workmanship in his experience, but no less impressive for that difference.  It was clear that every part of the sword was made with particular care, as if its makers intended it for some grand ceremonial use, or a particularly distinguished client.  Tarn held it close to his face, inspecting it.  On the base of the blade, right below the hilt, he found a small Dwarven inscription:

\dwarvenInscription{kildir}

\emph{\kildir}.  There was no other word engraved there; no mention of the sword's creators, or the place or time in which it was made.  But what was there was enough: this was the treasure for which he had traveled to these hills.  ``This is it'' he said.

Tarn had many attractive and valuable possessions back home in \korbarthrond: his Omunkorb model of the mountain, his rings, his coins of gold and silver \ldots\ even his guard's uniform was beautiful in its own way: functional, handsome, and clean.  His hammer was finely decorated and he maintained it well.  Additionally, he had seen even more impressive works adorning the city's halls, being sold in merchant stalls, and decorating the home of his friend Orvi.  Despite these treasures, there was no doubt in Tarn's mind that this teal sword now in front of him was the most beautiful object that his hands had ever held.  It filled him with emotions: satisfaction that his quest was complete; wonder at the mystery of its origins and abilities; awe at its antiquity and its legend.  Tarn imagined displaying it in his quarters; showing it to his friends and peers; holding it aloft like that ancient hero in the song while imagining the great deeds of the past.  Reciting \emph{The Land of Sea} and bragging about the elves and the Tholkis and the challenges that he had overcome in order to win this prize.

``Well that was easy,'' Peter said, smiling.  ``Now what?''

``We should explore further, and investigate what this Mothzam Redbeard is up to,'' suggested Circe.

``All I needed was the sword,'' Tarn said.  ``I know you bear a grudge against these dwarves, and I do not think ill of that.  But it is beyond the purpose of this quest.''

``I am of course happy that you have got what you needed for your city, my friend, but I have more to pursue here.  I am not on a mindless vendetta; we left the civilians unhurt, by my own art, and so I hope you can accept that I have made peace with the continued existence of these dwarves.  I do not begrudge them their lives, they who are merely living and working with no possible knowledge of any greater plan.  No, my interest lies with their leader, and his purpose, and whether there is some nefarious undertaking that must be discovered and stopped.''

Even when disagreeable and defensive, Tarn noted, the elf managed to maintain control over herself, speaking steadily and without excess emotion. ``Is there evidence that has made you suspicious of him?''  Tarn asked.

``Only the evil I have observed throughout my life.  I saw them raze my village as a child.  I have warred with them for decades.  The wood elves of Treeville do not intend to prosecute or investigate the Tholkis, but my dark-elf blood still flows hot with these violent memories.  It is my personal suspicion, and I will carry it with me until I am certain that there is no greater evil being perpetrated here.''

Peter decided to interrupt, for fear that his friends were becoming belligerent with each other.  ``Tarn, do you know how to use the sword to help Orehome?''

\emph{Orehome}.  From the moment he had laid eyes on the sword, Tarn had given not a single thought to his home city or the problem it had---the problem that had sent him here in the first place.  He was so worked up in his possession of \kildir that, yet again, he had forgotten his purpose.








Peter: do you know how to use it?
Tarn: oops; better go see the leader