\chapter{Clarity}

The fortress was cool inside, with a breeze gently passing them from deeper within the hill.  The main hall went straight forward, into the hill.  It smelled damp, and old.  Tarn guessed that this was only the surface level, and that below it there was much more to be seen.

Once inside the entrance he looked to the left, as advised by that divine voice, and saw a short passage leading to a wooden door.  He led the party down the hall, with Peter and Bes ducking under beams and taking care not to hit their heads on the dwarf-built ceiling.  Opening the door, Tarn found a small study, just as the voice had described.  Thankfully the room was unoccupied.  It contained many books, tablets and scrolls, as well as a number of maps hanging on the walls with varying amounts of annotation and sketches upon them.  There was a desk too, cluttered with documents and opened books, and samples of rock and metal that appeared to be under study.

On a shelf on the far wall, Tarn saw a glint of light and the shiny surface of a polished, teal-coloured metal.  His heart began to beat faster as he walked towards the shelf.  The object was under a pile of papers, which Tarn carefully removed and placed on the floor by his feet.  He reached up and grabbed the metal object.  It was a longsword.

``Is that it?'' Bes asked quietly, reverently.  ``\kildir?''

Tarn beheld it.  It was indeed teal colored; polished but not very sharp---not dulled through use, but rather as if it had been made blunt intentionally.  It was fairly long for a dwarf, reaching from about his waist to the floor.  His combat-trained arm could hold it up without much trouble; whatever this alloy was, it didn't seem as heavy as a bronze or iron, and Tarn doubted it would perform well in combat, except against an unarmed and unarmoured opponent.  The grip was not wrapped in leather; Tarn lamented the lack of care if it had deteriorated long ago and never been replaced.  There was a pattern finely engraved on the blade, a motif of water drops and flowing lines---different than the straight edges and geometric patterns common to dwarf workmanship in his experience, but no less impressive for that difference.  The crossguard was intricate and beveled.  The pommel bore a design of liquid dripping in all directions.  It was clear that every part of the sword was made with particular care, as if its makers intended it for some grand ceremonial use, or for a particularly distinguished client.  Tarn held it close to his face, inspecting it.  On the base of the blade, right below the hilt, he found a small Dwarven inscription:

\dwarvenInscription{kildir}

\emph{\kildir}.  ``Clarity'' in the tongue of Men.  There was no other word engraved there; no mention of the sword's creators, or the place or time in which it was made.  But what was there was enough: this was the treasure for which he had traveled to these hills.  ``This is it'' he said.

Tarn had many attractive and valuable possessions back home in \korbarthrond: his Omunkorb model of the mountain, his rings, his coins of gold and silver \ldots\ even his guard's uniform was beautiful in its own way: functional, handsome, and clean.  His hammer was finely decorated and he maintained it well.  Additionally, he had seen even more impressive works adorning the city's halls, being sold in merchant stalls, and decorating the home of his friend Orvi.  Despite these treasures, there was no doubt in Tarn's mind that this teal sword now in front of him was the most beautiful object that his hands had ever held.  It filled him with emotions: satisfaction that his quest was complete; wonder at the mystery of its origins and abilities; awe at its antiquity and its legend.  Tarn imagined displaying it in his quarters; showing it to his friends and peers; holding it aloft like that ancient hero in the song while imagining the great deeds of the past.  Reciting \emph{The Land of Sea} and bragging about the elves and the Tholkis and the challenges that he had overcome in order to win this prize.

``Well that was easy,'' Peter said, smiling.  ``Now what?''

``We should explore further, and investigate what Mothzam Redbeard is up to,'' suggested Bes.

``All I needed was the sword,'' Tarn said.  ``I know you bear a grudge against these dwarves, and I do not think ill of that.  But it is beyond the purpose of this quest.''

``I am of course happy that you have got what you needed for your city, my friend, but I have more to pursue here.  I am not on a mindless vendetta; we left the civilians unhurt, by my own art, and so I hope you can accept that I have made peace with the continued existence of these dwarves.  I do not begrudge them their lives, they who are merely living and working with no possible knowledge of any greater plan.  No, my interest lies with their leader, and his purpose, and whether there is some nefarious undertaking that must be discovered and stopped.''

Even when disagreeable and defensive, Tarn noted, the elf managed to maintain control over herself, speaking steadily and without excess emotion. ``Is there any evidence that has made you suspicious of him?''  Tarn asked.

``Only the evil I have observed throughout my life.  I saw them raze my village as a child.  I have warred with them for decades.  The wood elves of \inarthonor\ do not intend to prosecute or investigate the Tholkis, but my dark-elf blood still flows hot with these violent memories.  It is my personal suspicion, and I will carry it with me until I am certain that there is no greater evil being perpetrated here.''

Peter decided to interrupt, for fear that his friends were becoming belligerent with each other.  ``Tarn, do you know how to use the sword to help Orehome?''

\emph{Orehome}.  From the moment he had laid eyes on the sword, Tarn had given not a single thought to his home city or the problem it had---the problem that had sent him here in the first place.  Exactly as had happened in the Silverdale library when first hearing about the sword, he was now so worked up in his possession of \kildir\ that he had forgotten his purpose.

But that possession had immense power over Tarn.  In this moment he felt complete satisfaction and victory.  If he were to leave now, travel home to \korbarthrond\ and never actually use the sword's purported magic, then he would still feel like he had accomplished a lifetime's worth of challenges.  Such was the greatness of this artefact.

Despite this temptation, Tarn did admit to himself that he did not know how to use \kildir\ to restore \korbarthrond's water supply.  He quickly remembered that \arilor\ had doubted that even \mothzam\ \driktur\ would have that knowledge, and if he did not then pursuing him would serve only to put Tarn's party in danger, with no benefit seen.  After raiding \mothzam's study, killing his guards, and stealing \kildir, any confrontation would surely be violent.

Instead, Tarn could leave now, and travel back home.  Maybe he could figure out the incantation or ritual that would awaken the power in the sword.  It was unlikely, but that may be his only course of action regardless of what else eventuated within this hill fortress.  There was probably no need to start a fight when he could walk away now with just as much knowledge as if he were to defeat \mothzam.  Why risk his life?

Indeed, why should he risk his life at all?  He was no hero, only a common guard.  Why should the fate of \korbarthrond\ rest on his small shoulders?  King Athzad had invented a number of other solutions to the water crisis, and one of them could bear fruit.  The city could have been saved already, for all he knew, with some new underground river being welled, or a channel being cut through to the mountaintop streams.  Tarn was just a lone dwarf, far from home entertaining delusions that he could be the saviour of his city.  Why not simply face reality, accept that he alone could not make a difference, and go home now with his life and his treasure sword?

He picked up the sword, holding the blade in both hands, and looked at it.  Beauty itself, in metallic form.  The flat bevels of the blade were polished to a mirror finish, and in the teal depths he saw the reflection of his own contemplative eyes.  As he turned the blade he saw his nose, his satisfied mouth, his beard, and \ldots\ the silver necklace given to him as a gift by Peter on the day they met in Silverdale.  The necklace that was intended to remind Tarn to follow his conscience, to seek the path of the good, to live up to his potential.  But Tarn was a Dwarf; would it not satisfy his potential  if he were to possess this sword of great craftsmanship and beauty and history?

Then Tarn recalled the \emph{Song of the Giants}.  It was easy to focus on `finding his grace' in the ownership of this great metal artefact.  But there was more in the song---it was nominally about the Giants, after all, and not the Dwarves.  The Giants, whose greed for stone became their downfall, yet here was a dwarf, descended from the Giants, exercising that same greed.  It was a mere children's song, admittedly, but its themes were deeply embedded in Tarn, and likewise probably in most dwarves.  And there was more: it was in \emph{conquering} stone, not possessing it, in which Dwarves supposedly became fully manifest.  \korbarthrond, an entire city carved into a mountain, was an enormous embodiment of that conquest.  And it was under threat.  Tarn took a deep breath upon realising this.  The city was a greater expression of Dwarven life than a single person ever could be, and it was worth saving.  It was worth giving \korbarthrond\ the best chance possible.  \emph{That} was the path to meeting his potential because that was the path to enabling a greater potential, a future, for thousands of others.

Tarn finally answered Peter.  ``I do not know how to use it,''  and, turning to look at Bes, ``but \mothzam\ \driktur\ might.  And while there is any chance that I can learn those secrets from him, I will desire to confront him.''

Bes smiled in a quiet victory.  Peter nodded, having already made his peace with the upcoming conflict.  





