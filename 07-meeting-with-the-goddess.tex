\chapter{The Village of Tholk\=unrond}
%\chapter{The Gates of Exile}

After a short rest and a hot meal, Tarn, Peter and Circe cautiously continued their journey south.  If the scouts had come this far to patrol, then Tholk\=unrond must have been quite close.  Eventually the plains began to give way to hills, and one of those hills finally showed a sign of civilisation, featuring a redoubt.  There were no guards on watch---perhaps the scouting party they'd killed had been posted there.

They went on carefully.  In the distance they could see smoke, meaning some of the other hills were actually occupied.  When they got close enough to see the buildings themselves, they stopped, and found a high sheltered place, to hide and observe.  The area seemed to be the outskirts of a small village, with farmhouses and workshops built into the small hills, and fields of crops in the valleys between them.

Tarn was a mountain dwarf, and the natural way of things to him was that dwarves carve expansive fortresses deep into mountains.  He had heard about hill dwarves before, but had no first-hand experience with their culture, and had never even known that there were hill dwarves living near Korbarthrond.  But here they were.  And their approach to building was quite alien to Tarn.

Most structures seemed to be half under the ground, and half above; half excavated and half constructed.  A balance between the Human approach and what Tarn saw as the more conventional Dwarven approach.  It made some sense: these hills were significantly smaller than Korbarthrond's mountain.  They simply didn't have the space for a large complex.  But these were still dwarves---still have been driven by that primal urge; \emph{they longed for their caves, to be girded with stone \ldots\ to carve out their own space}, as the \emph{Song of the Giants} had put it.  So this was a compromise.  The town itself could only be described as being above ground, with certain commons, wells, pastures and fields of crops lying completely under the sun.  It was the buildings themselves that were mostly half-buried in the hills, with a doorway poking out here, a window there, an annex constructed of wood or stone, or a high tower popping out the top.  Familiar, but different.

One significant disadvantage to the hill-dwarf approach to city building was that it was much harder to defend.  When an entire city lay in the core of a mountain, a few gates and tunnels were the only way in, and thus the only parts requiring defense.  And even then, they were often narrow enough that a few well-placed barricades and traps would suffice to keep invaders out---at least, for long enough that soldiers could arrive from within the city.  This town, conversely, was spread out and open.  Each individual homeowner might be able to hold off those trying to enter his hill-house, but the bulk of the town was vulnerable, and even if one house were safe, others may not be.  Tarn felt deeply grateful that Korbarthrond was a a mountain fortress instead of a town of hills.

Towards the back of the village was a large hill with a kind of bunker carved into it.  It clearly had underground levels, and had a guard standing out the front.  Tarn supposed that this building was the headquarters of the gang's leader.

It was late morning now, and the village had come to life.  Farmers were working their fields, hunters had left for nearby woods, and craftsmen were busy in their workshops making fine and useful wares.  An engraver worked on a constructed wall jutting out of a hillside, decorating it with a pleasant geometric pattern.  The large central well saw dwarves coming and going constantly with buckets and pots.

``They're civilians,'' Tarn said, after they had been watching Tholk\=unrond for a while.  ``We should let them be.''

``If the gang is up to some dark purpose, then so are these people,'' Circe retorted.  ``Just because they grow food instead of kill, it doesn't make them less culpable for the actions of the group as a whole.''

Tarn bristled at this.  He wanted this to be as peaceful as possible.  Circe seemed to be looking for excuses to take out her revenge on the dwarves.  ``If the leader is guilty of something,'' offered Tarn, ``these villagers may not know about it.''

``Then why would they live here, in this tiny village in the middle of nowhere, scraping together a living with no resources, no allies, and just a well for water?''

Tarn had no answer.

``Anyway, soon they will realise that the scouts are missing,'' Circe snapped back.  Tarn didn't want to admit it, but she was right.  If news spread that the guards were missing---or worse yet, if their bodies were discovered---then there may be additional guards, or the whole village may mobilise,.  These civilians may not stay civilians very long.

Tarn glanced over to Peter.  The cleric had been listening to the conversation with a mournful look on his face, dreading yet more combat.  Still shaken by the early morning's ordeal, he said ``I almost died last night.  I may be part of the Order of Light, but I'm not ready to actually \emph{see} the Light.''  Then, after a pause, ``I though I was okay with dying.  Until I felt death come close.  Now I'm just scared.''

``There's nothing wrong with being afraid of death, Peter,'' said Tarn reassuringly.  ``You've just never had to really think about it before.''

``But I \emph{have} thought about it.  At least at an intellectual level.  I believe that when I die, as a good person I will be subsumed into a permanent state of light and goodness.  But if I'm afraid of dying, when it really matters, does that mean I don't have a true faith?''

Tarn smiled at him.  ``It means you're a normal person with normal feelings.  The depth of battle pushes every person to his limits.  I'm a soldier---well, a guard.  I'm very firm in my knowledge that I would give my life for my city.  Part of that is my training, but part is inherent to me, and my training simply made me cognisant of it.  There's no question in my mind.  But when blood is high, and there's a real fear in the air, and your friends are falling around you, it's only natural for the heart to run away from the head and start asking questions.  It doesn't mean that your principles aren't sincere.  It simply means that your emotions aren't a slave to them.

``That night in the mud, you told me to just follow my conscience; do you remember?  Look at it that way.  Your fear tempts you to abandon your faith, but your conscience tells you to persevere and not let the heat of the moment distract you from what you truly believe.  Acknowledge that you might \emph{feel} otherwise, and remember that such feelings do pass eventually.  Your feelings don't define who you are.  Your choices do.''

``\ldots\ Are you sure you're not a cleric?'' Peter asked, smiling weakly.

``Just a fighter.  The wisdom is nothing new; I just used your words to describe it.

``But you're not a fighter, Peter.  You did fight well, but fighting and killing are not your vocation.  I won't think any less of you if you want to turn back here.  You've already helped us beyond measure.''

``That's kind of you to say, Tarn, but I gave you my word that I'd see this through with you, and the fear of death pales in comparison the fear of breaking my word.''  He forced a chuckle.  ``Would you mind praying with me?''

The dwarf's face fell.  ``Uhhm \ldots\ I don't mind talking about theology with you, but I'm not a believer.''

``How many times have you seen me heal a wound, or command some other supernatural power?  Can you truly say that you don't believe in those?''

``I don't deny that you have performed some impressive magic.  But so has Circe, and she isn't using her feats as a chance to proselytise''---Tarn looked over at her---``right?  For \emph{you}, it's a discipline to be learned and not a religion to be followed.''

``I suppose that's true.'' answered the elf.

Peter responded, ``I don't understand the pyromancer's craft, but it must somehow involve drawing energy from the hidden parts of the world.  My `magic' is similar, but I draw on the goodness of that \emph{outside} the world; its creator.  I request rather than command.  And in my request, I'm not trying to convert you to a religion or teach you how to yourself perform healing magic.  I am struggling with my own feelings and convictions. I ask that you pray with me, to lend me your support during this personal difficulty of mine.''

``\ldots\ That sounds reasonable,'' Tarn answered hesitatedly.

``Additionally, if you open your heart to the good, you may curry favour with the Light.  I believe it would make it easier for me to heal you.''

``Alright, priest,'' Tarn sighed, ``I'll try praying.  Tell me what to do.''

Peter smiled at his friend.  ``First of all, if at any point you start to find this disagreeable, please feel free to stop.''  Tarn nodded.  ``Take off your necklace, and hold it in your hand.  Kneel down.  Now close your eyes.''  Tarn did so.  ``Repeat these affirmations to yourself: \emph{I want to be a good person.  I want to reach my potential.  I want to build, and not destroy.  I want to elevate, and not subvert.  I ask for guidance to follow this path.}''

Tarn didn't find that any of those statements conflicted with his personal beliefs, so he humoured his friend.  What dwarf doesn't want to build, after all?  And wouldn't every person consider himself to be `good'?  He repeated the statements quietly to himself.

Suddenly, he could sense a bright light shining through his close eyelids, so he opened his eyes.  He could see nothing: pure white, as if the sun had taken over the entire sky and the world beneath it.  But the brightness didn't hurt his eyes like the sun would.  It was simple, the most simple thing, and yet it was beautiful, even to Tarn's Dwarven eyes that so valued detailed and complex work.  He looked down at his hands, and couldn't see them.  He opened his mouth.  ``What is this?'' he asked, and he heard his own voice.  He may not have been able to see anything, but at least he could still speak.

As if in response, he heard another voice, not his own.  It said ``Hello Tarn,''  in Dwarven, his native tongue.  The voice had a familiar quality to it: that deep majesty that had echoed through Peter's voice for that brief moment during the fight against the scouts.  Clear and loud, resonating.  But this was not Peter's voice.  It was a female voice; penetrating, soothing.  Transcendent.  Perfect.  The speech made Tarn feel warm, satisfied, as if each word were gently and firmly embracing him.

Tarn repeated his question, but in a gentler tone; inquiring rather than interrogating.  ``What is this?  Who are you?''

``I am not of this world, but I have been watching over you,'' the voice answered.

``Am I still praying?  Are you Peter's god---`the Light'?  I seem to be doing quite well for someone who's never prayed before!''

``The manner of prayer is inconsequential.  Your intention was clear enough to demonstrate that you are on the right path.''

``And what path is that?''

``You want to be a moral person.  To be brave.  These are commendable goals.  Yet it requires effort to reach them; to do what you believe is right; to fulfill your destiny.''

``Destiny?'' Tarn was fine with being called `moral' and `commendable', but he was sure that he was not driven by some `destiny'.  He was a dwarf!  He forged his own path, created his own life.

``Right now, your destiny is tied to your quest to save Korbarthrond.  Above all else you want the sword, K\=\i{}ldir.''  At this, Tarn looked down, withdrawing physically, ashamed.  Ashamed that that was his most profound motivation, and ashamed that somebody knew else knew it.  The voice continued. ``This is not inherently bad.  Dwarves make, collect, and admire beautiful things.  It is a force of good in the world, as it encourages beautiful things to come into being.

``Korbarthrond needs your help, and K\=\i{}ldir may save her.  But it cannot do this if you keep it to yourself, for yourself.  You must make the choice between your city and your desire to possess.  I cannot make this choice for you.  What I \emph{can} do is assist you in acquiring the sword.

``The leader of the Tholkis is a dwarf named Mothz\=am Drikt\=ur. You have guessed that he is in the fortress here.  Right inside the front gate is a passage left, to a study.  Within that room lies the sword you seek.''

``Thank you \ldots\ I don't know what to say,'' Tarn admitted.

``No answer is necessary.  I merely suggest that you make the right choices, to be a source of goodness in the world.''

With that, the white faded and he could see again, more and more as his eyes adjusted to the relative darkness of the real-world afternoon: first the sun took shape, then the land and village that stretched out beneath him, then his companions.

Tarn turned to Peter. ``I believe I have just communed with your goddess.''.

``Goddess?  What happened?''

``I saw bright white, everywhere.  I couldn't move, or perhaps just couldn't see myself move. There was a voice, like the enchanted voice you used.  She told me to bring about goodness in the world, and to take the sword to save Orehome.''

Peter was surprised.  

Peter was astounded.  He himself had certainly been influenced by the Light, having received signs and signals about what was desired of him, but he had never experienced anything like what Tarn was describing, something so \ldots\ real, unambiguous, direct.  In all his studies he had never learned of the Light itself speaking to a person, even to those first races that walked upon the newly created world.  So it couldn't have been the Light itself.  Religious tradition spoke of angels who marched under the banner of Light in the Invisible War, sometimes intervening in worldly affairs.  Perhaps it was such an angel who spoke to Tarn?

He tried to make sense of his feelings---happiness for his friend; jealousy that the Light seemed to favour a lay guardsman over a life-long cleric like himself; reassurance that his faith in Tarn's quest was well-placed.

``There's more,'' said Tarn.  ``The voice told me where to find K\=\i{}ldir.''

Circe looked shocked.  ``Are you sure?'' she asked.

``Yes.  She was very specific.  Apparently there is a passage left, to a library, where the sword can be found.  Oh, and the leader's name is Mothz\=am Drik't\=ur''---he translated the surname---``Redbeard.''

``We should raid the bunker now,'' Circe said anxiously.  ``We know the sword is in there.  Let's go before the village discovers those dead scouts.''

Tarn thought about their options.  ``You are right about those scouts.  We should infiltrate sooner, rather than later.''  They had been planning to wait until the cover of darkness before attempting to enter the bunker, but that may have been too late.  ``How do we get from here to there in the light of day, without being seen?''

``A distraction!'' chirped Circe.  ``I could create a fire tornado out to the west, that might distract all of the villagers.  Or light one of their buildings on fire so that they all run there to help extinguish it.''

As much as Tarn wanted to avoid harming civilians, something like this really was a lesser aggression.  A large fight would do much more harm to the village, while also creating significant danger for himself and his companions.  ``A building,'' he nodded.  ``But nothing that will cause too much grief \ldots''

``How about that house?'' Circe offered, pointing at a small hill.  It was about halfway between the party and the fortress, but quite far off to the side; drawing the villagers there might clear enough of a path.  The house itself was mostly inside the hill, but an entrance porch stuck out, made of wood with a thatched roof.  The family who lived there would still sleep under a roof that night, even if the porch were completely burned away.

``Very well,'' Tarn said. ``Is everybody ready?''  Circe and Peter nodded.  The dark elf then faced the house with staff raised, muttered a spell, and suddenly the walls and roof of the porch were ablaze.  There were shouts from below; villagers ran in many different directions, but mostly towards the house or towards the well to get buckets of water.  The three companions quickly came down from their hill and, slipping between buildings and staying out of sight of the commotion, ran towards the fortress.

The guard in front of the doorway were the only dwarf in the village that noticed them.  He was alert immediately, spear in hand and an angry expression on his face.  Tarn faced him with shield and hammer drawn, while the other two stayed behind him.  The guard ran for Tarn, who held his shield out poised for the strike.  Just as the spear was close enough to strike, the guard suddenly dropped to the ground yelping and rolling as more flames from Circe's staff consumed him.  Tarn dutifully ended the torment with his hammer, and the three companions entered the doorway.